
\section{Threat Model \& Assumptions}

\wajih{This threat model does not talk about privacy. Please add that aspect.}
To develop an all-encompassing model for addressing cybersecurity threats, we base our approach on certain hypotheses regarding the attacker's tactics and the system's features. We presume that the attackers operate discreetly, camouflaging their harmful activities within the normal data flow. This strategy significantly complicates the process of distinguishing their actions from legitimate operations. Additionally, these attackers might employ previously unknown vulnerabilities, or zero-day exploits, indicating a lack of established attack patterns for reference during training. This absence of historical data presents a considerable obstacle in formulating robust detection methods.

\wajih{please do not use words like posit and deeds they do not sound academic}

Within our model, we posit that to execute their harmful deeds, attackers inevitably create detectable traces in the system's logs, distinct from standard operations. These traces are expected to reveal differences in the configuration of the attacker's node compared to authentic nodes with similar classifications. Through a detailed examination of the network's graph structure and the attributes of individual nodes, we aim to pinpoint anomalies that align with the attacker's maneuvers and monitor them persistently. Echoing the principles found in preceding data provenance research~\cite{nodoze2019,priotracker2018,mzx2016,bates2017transparent}, we operate under the assumption that our data provenance system accurately and comprehensively documents all system activities and modifications. Furthermore, we trust in the unbroken integrity of our audit logs and the embedding recycling database, safeguarded by existing tamper-proof storage solutions~\cite{paccagnella2020custos,hardlog}. This trust ensures that the provenance graphs created by \Sys are dependable and serve as effective tools for identifying and dissecting cybersecurity threats.