\section{Adversarial Attacks Analysis}
\label{sec:adversarial}

%\wajih{can we make subsections here.}

\Sys may be vulnerable to various types of adversarial attacks, including gradient-based attacks, model poisoning, and inference attacks. Inference attacks are already discussed in Section~\ref{privacy}.

\PP{Gradient Attacks} Gradient-based adversarial attacks~\cite{chakraborty2021survey} typically require white-box access to the target machine learning model, including its parameters. This necessity often renders them impractical for real-world applications. In contrast, black-box attacks, which utilize iterative, query-based techniques, tend to be more detectable and complex to implement due to their conspicuous nature. Such attacks are feasible if an attacker manages to compromise a client machine. However, during the operational phase, a compromised client cannot affect other clients because they are working independently. Several existing defenses can be employed during model training to enhance the system's resilience against these attacks. Adversarial training~\cite{tramer2019adversarial} is one effective strategy, wherein the model is trained with perturbed input data to increase its robustness to such attacks.

\PP{Poisoning Attacks} During the training phase, poisoning attacks executed by malicious actors may introduce corrupt weights to compromise the global model~\cite{jagielski2018manipulating}. To improve \Sys's resilience against such threats, several defensive methods can be used. Among these, advanced model aggregation methods, such as Multi-Krum~\cite{munoz2019byzantine}, can be particularly effective. This method employs clustering techniques on the central server to identify anomalous updates during model aggregation. Consequently, outlier updates are removed, enhancing the system's robustness against poisoning.

\PP{Mimcry Attacks} \wajih{What about Akul's mimcry attack on PIDS? Also cite this paper https://www.usenix.org/conference/usenixsecurity23/presentation/mukherjee. }
