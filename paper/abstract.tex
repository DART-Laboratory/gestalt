\begin{abstract}
    Enterprises increasingly rely on Intrusion Detection Systems (IDS) to detect malicious threats through log analysis. However, centralizing logs, which often contain sensitive data (e.g., URLs), raises serious privacy and scalability concerns. We present \Sys, the first privacy-preserving Provenance-based IDS (\pids) that integrates Federated Learning (FL) with graph representation learning to match centralized detection accuracy while preserving privacy and improving scalability. Building \Sys is non-trivial due to challenges in federating graph-based models across clients with heterogeneous logs, inconsistent semantic encodings, and temporally misaligned data. To address these challenges, \Sys introduces a novel process entity categorization-based ensemble, where specialized submodels learn distinct system behaviors and avoid aggregation errors. To enable privacy-preserving semantic alignment, \Sys designs a dual-server harmonization framework: one server issues encryption keys, and the other aggregates encrypted embeddings without accessing sensitive tokens. To remain robust to temporal misalignment across clients, \Sys employs inductive GNNs that eliminate the need for synchronized timestamps. Evaluations on DARPA datasets show that \Sys matches the detection accuracy of state-of-the-art PIDS, reduces network communication costs by 170$\times$, processes datasets in minutes rather than hours, and remains robust against adversarial threats.
    \end{abstract}
    
    

% Data provenance transforms \logs logs into detailed provenance graphs, offering a better understanding of host activities. When combined with Graph Neural Networks (GNNs) and FL, it becomes a powerful technique for differentiating benign from malicious behaviors.

% We have designed a multi-server architecture and a robust encryption scheme to preserve the privacy of important user log attributes. To address the challenges of non-IID data distributions, heterogeneous, and imbalanced clients, we implement a sophisticated GNN learning framework personalized at the system entity level. We also present a detailed analysis of our system's resilience against adversarial attacks. Extensive evaluation of our system on real-world datasets from DARPA demonstrates that it achieves state-of-the-art detection performance while being highly scalable and privacy-preserving.