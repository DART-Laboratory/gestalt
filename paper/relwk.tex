\section{Related work}
\label{s:relwk}

In this section, we will discuss existing systems related to the domain of provenance-based intrusion detection.

\wajih{ Make sure to cite this paper: https://www.usenix.org/system/files/sec23winter-prepub-490-jia.pdf}

\PP{Machine Learning based PIDS} Many existing systems leverage machine learning techniques for threat detection. ProvDetector~\cite{provdetector2020} utilizes the Doc2Vec~\cite{le2014distributed} model to encode attack paths in provenance graphs into embeddings for outlier detection. Attack2Vec~\cite{shen2019attack2vec} employs a temporally aware word-to-embedding encoding scheme to identify attack entities. In the realm of network anomaly detection, DeepAid~\cite{deepaid} utilizes deep neural networks to differentiate anomalous traffic. DISTDET~\cite{dong2023distdet}, a host intrusion detection system, detects advanced persistent threats (APTs) using hierarchical system event trees. ProGrapher~\cite{yangprographer} generates provenance graph embeddings for anomalous graph identification by integrating Graph2Vec~\cite{narayanan2017graph2vec} and TextRCNN~\cite{lai2015recurrent} models. StreamSpot~\cite{streamspot}, another graph-level detector, constructs benign models using various graph features and detects anomalies through clustering algorithms. Furthermore, \unicorn~\cite{han2020unicorn} employs graph kernels for graph-level threat detection. Other IDS systems~\cite{aljawarneh2018anomaly, maseer2021benchmarking, gyanchandani2012taxonomy,atlas} also utilize diverse embedding generation techniques. Additionally, studies such as \cite{zolkipli2011approach, chakkaravarthy2019survey, isohara2011kernel} focus on malware detection. Techniques like DeepLog~\cite{deeplog2017} directly process logs using recurrent neural networks. SIGL~\cite{sigl} specifically targets the detection of malicious software installations, while Euler~\cite{king2022euler} employs both GNN and RNN models to detect lateral movements.

\PP{Rules-based IDS} Another category of PIDS concentrates on utilizing predefined rules to detect malicious entities. Key examples include Holmes~\cite{holmes2019}, Rapsheet~\cite{rapsheet2020}, and Poirot~\cite{poirot2019}. These approaches leverage insights from Advanced Persistent Threats (APTs) to formulate their rule bases. In comparison to IDS systems based on machine learning, rule-based IDSs tend to generate fewer false positives. Nevertheless, a significant limitation of these systems is their inability to identify threats featuring novel attack signatures. Additionally, they often necessitate the expertise of skilled security professionals for the development of their rule sets.

\PP{Federating Learning in Threat Detection}
\wajih{Do not redefine the PIDS and FL acronyms again and again as you did below}
Currently, there are few Intrusion Detection Systems (PIDS) that utilize the principles of Federated Learning (FL). The majority of research focuses on Network Intrusion Detection. \cite{man2021intelligent} proposes a technique for threat detection in IoT devices using FL, while \cite{friha20232df} introduces a differentially private detection system tailored for industrial IoT environments. Additionally, \cite{li2023efficient} presents an efficient network intrusion detection system. \cite{yang2023privatefl} addresses the challenge of heterogeneity introduced by differential privacy in federated learning systems. Furthermore, \cite{guo2023new} discusses the impact of non-IID data on federated learning for intrusion detection, proposing methods to mitigate this issue. Lastly, \cite{chaabene2023privacy} describes another FL-based system for IoT intrusion detection.

\PP{Privacy-preserving FL}

\wajih{papers to cite:}
https://www.usenix.org/system/files/usenixsecurity23-yang-yuchen.pdf


https://www.ndss-symposium.org/ndss-paper/fp-fed-privacy-preserving-federated-detection-of-browser-fingerprinting/

https://dl.acm.org/doi/10.1145/3560830.3563729

https://www.ndss-symposium.org/ndss-paper/poseidon-privacy-preserving-federated-neural-network-learning/

https://ieeexplore.ieee.org/document/10091486


https://arxiv.org/abs/2310.20552

https://dl.acm.org/doi/10.1145/3624017


\wajih{You need to go through these papers, especially their threat model and evaluation section. Tell me what metrics are they using to prove that their FL is privacy-preserving. And you need to tell me why we can or cannot apply to our system.}
