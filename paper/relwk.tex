\section{Related work}
\label{s:relwk}
In Section~\ref{sec:motivation}, we described the challenges with existing provenance-based IDSes that \Sys addresses, and complement the discussion on related work here.

\PP{Specification-based IDS} Specification-based IDS methods such as Holmes~\cite{holmes2019}, Rapsheet~\cite{rapsheet2020}, and Poirot~\cite{poirot2019} detect deviations from normal behaviors using specification rules on provenance graphs. Although these approaches effectively reduce false positives, they risk being circumvented by new attacks and require continuous updates to address the evolving threat landscape. In contrast to these systems, \Sys is a more scalable and robust solution, capable of detecting previously unseen threats by leveraging rich semantic and contextual information from provenance graphs.

\PP{Embedding-based IDS} Embedding techniques are widely used for log analysis tasks, such as IDS~\cite{aljawarneh2018anomaly, maseer2021benchmarking, gyanchandani2012taxonomy,atlas} and malware identification~\cite{zolkipli2011approach, chakkaravarthy2019survey, isohara2011kernel}. They often employ ML models, such as neural networks and n-grams to transform logs into vector forms. Examples include DeepLog using LSTM~\cite{deeplog2017}, ProvDetector applying Doc2Vec~\cite{provdetector2020, le2014distributed}, and Attack2Vec leveraging a temporal word-embedding model~\cite{shen2019attack2vec}. Euler~\cite{king2022euler} uses \gnnshort and RNN embeddings to detect lateral movements. SIGL~\cite{sigl} focuses solely on detecting malicious software installations via deep learning and suffers from the limitations of graph-level granularity detection. Moreover, SIGL's evaluation is based on a small dataset of normal/malicious software installations, making it challenging to scale to large provenance graphs. DeepAid~\cite{deepaid} uses deep neural networks to detect anomalies in network traffic. Different from these systems, \Sys is a host-based IDS blends \gnnshort with rich semantic word embeddings from system logs and stores training-phase embeddings for real-time APT detection.

\PP{Provenance-based Investigation.} Hassan et al.~\cite{winnower2018} utilized grammatical inference over provenance graphs to expedite system intrusion investigations. Pasquier et al.~\cite{pasquier2018runtime} introduced CamQuery, a real-time provenance analysis framework. Furthermore, existing systems,such as DEPCOMM~\cite{xu2022depcomm}, DEPIMPACT~\cite{fang2022back}, Watson~\cite{watson}, NoDoze~\cite{nodoze2019}, Palantir~\cite{zeng2022palantir}, Deepcase~\cite{van2022deepcase}, SAQL~\cite{217496}, OmegaLog~\cite{omegalog}, C2SR~\cite{kwon2021c}, Dossier~\cite{dossier}, and Atlas~\cite{atlas}, facilitate attack investigations and incident response. These existing systems are orthogonal to our research direction as they are not designed to detect APT attacks. They require initial attack symptoms from intrusion detectors to initiate investigations.

