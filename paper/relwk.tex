\section{Related work}
\label{s:relwk}
In Section~\ref{sec:motivation}, we have outlined the limitations of current provenance-based intrusion detection systems (IDS) that \Sys seeks to overcome. This section further expands on the discussion of related work.

\PP{Machine Learning based IDS} Embedding strategies are prevalent in log analysis for IDS tasks~\cite{aljawarneh2018anomaly, maseer2021benchmarking, gyanchandani2012taxonomy,atlas} and malware detection~\cite{zolkipli2011approach, chakkaravarthy2019survey, isohara2011kernel}. These methods typically use machine learning techniques, like neural networks and n-grams, to convert logs into vectors. For instance, DeepLog employs LSTM~\cite{deeplog2017}, ProvDetector uses Doc2Vec~\cite{provdetector2020, le2014distributed}, and Attack2Vec utilizes a temporal word-embedding model~\cite{shen2019attack2vec}. Euler~\cite{king2022euler} combines \gnnshort with RNN embeddings for lateral movement detection. SIGL~\cite{sigl} is focused on detecting malicious software installations using deep learning but is limited by its graph-level granularity and relies on a small dataset, making it less effective for large-scale provenance graphs. DeepAid~\cite{deepaid} employs deep neural networks for network traffic anomaly detection. \Sys, in contrast, is a host-based IDS that integrates \gnnshort with comprehensive semantic word embeddings from system logs, preserving embeddings from the training phase for efficient real-time APT detection.

\PP{Rules-based IDS} Approaches like Holmes~\cite{holmes2019}, Rapsheet~\cite{rapsheet2020}, and Poirot~\cite{poirot2019} in specification-based IDS rely on predefined rules applied to provenance graphs to spot abnormal activities. These methods are successful in minimizing false positives but fall short against novel attacks, necessitating regular updates to keep up with new threats. \Sys, however, offers a scalable and effective alternative, detecting new types of threats by harnessing the detailed semantic and contextual data in provenance graphs.

