\section{Introduction}
\label{s:intro}

Host-based Intrusion Detection System (HIDS) plays a crucial role in today's cybersecurity infrastructure. They monitor and analyze system logs to detect and counteract malicious activities. However, a traditional HIDS needs centralized storage and processing of these logs. Given that these logs might contain sensitive information, there's an increasing concern about user privacy and data leakage risks. In parallel to this, with the rise of Managed Security Service Providers (MSSPs), a shift in cybersecurity strategy is taking place. MSSPs are external entities that expertly manage the security of client businesses. Their services include collecting sensitive logs and then storing and analyzing them in the cloud for intrusion detection and response. Recent data~\cite{study-mssp} shows an increasing trend towards this model, with 70\% of businesses planning to use MSSPs soon. Although cost-effective and efficient, this method processes sensitive data from multiple enterprises in shared environments, raising significant data breach concerns~\cite{attack-mssp}.

In this context, the need for a privacy-preserving HIDS that can operate efficiently in an enterprise network is more pronounced than ever. Federated Learning (FL) emerges as a potential solution to this issue. FL enables decentralized machine learning models to be trained on distributed data sources without the need to share raw data. By applying FL to intrusion detection, privacy-preserving systems can effectively learn from the collective knowledge of multiple hosts in the enterprise while safeguarding individual users' data privacy. This approach is particularly suited for scenarios involving MSSPs where a diverse range of data sources must be considered without violating privacy.

Unfortunately, integrating FL into HIDS presents challenges, particularly regarding model accuracy~\cite{bonawitz2019towards}. A potential answer to these challenges is found in \textit{Data Provenance} analysis. Applied to system logs, this analysis translates into provenance graphs, which provide a detailed overview of host activities. These graphs allow for a deep understanding of the causality and support the development of more robust and accurate HIDS~\cite{provdetector2020,rapsheet2020,shadewatcher}. For instance, in Figure~\ref{fig:example:provenance}, a Firefox process downloads a malicious PDF file, leading to the activation of a keylogger. By using this graph, system defenders can reconstruct the whole attack story. In essence, the fusion of provenance graphs and FL offers an innovative direction for improving HIDS accuracy without compromising user privacy.