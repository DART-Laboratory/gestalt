\section{Introduction}
\label{s:intro}

Intrusion Detection Systems (IDSes) play an essential role in enterprise security strategies to counteract Advanced Persistent Threats (APTs). APTs are notably stealthy and persistent, exemplified by the significant system disruptions seen in attacks such as Solar Winds~\cite{solarwinds} and NotPetya~\cite{notpetya}. The effectiveness of IDSes hinges on their ability to accurately detect these threats, maintain low false positive rates, and operate with minimal resource consumption, ensuring system performance is not compromised.

\wajih{Give the workflow of MSSP and cite that organizations outsource their security. If you find any numbers on how many comapnies use MSSP and outsource their security operations that would be great. If you find any realworld attack examples on MSSP where the data was leaked that would be great as well. }

Recent advancements in enterprise security have seen the integration of data provenance in IDSes. By analyzing system (audit) logs and converting them into provenance graphs, these systems offer a comprehensive view of system execution. Provenance-based IDSes~\cite{streamspot,provdetector2020,wang2022threatrace,shadewatcher,yangprographer,han2020unicorn} have emerged as a potent solution, utilizing the rich context within audit logs to improve detection capabilities. Despite their promise, these systems face significant challenges in the complex landscape of enterprise security:

\begin{itemize} [leftmargin=*]
    \item[--] \textbf{Lack of Privacy Preservation:} Traditional provenance-based IDSes (PIDS) \wajih{citations?} rely on a centralized infrastructure, necessitating client machines to transmit their log data to a central server. \wajih{For example, Flash[] and Kairos[] send plain text logs to central server to train GNN blah blah...} This approach risks user privacy as audit logs encompass detailed records of system execution, including sensitive information and activity patterns.
    
    \item[--] \textbf{Excessive Network and Disk Overhead:} Audit logs can amass to gigabytes per day for each client, imposing considerable network costs on both users and organizations. The demand for extensive disk storage to accommodate these logs further complicates matters. Current PIDS models overlook these issues, affecting their viability for practical deployment. \wajih{Could you give specific numbers about network overhead and disk overhead that would be great.} \wajih{Again cite existing PIDS with names that would suffer from this issue.}
    
    \item[--] \textbf{Scalability Challenges:} The centralized model employed by existing PIDS presents scalability issues as the number of hosts within an organization grows. The centralization becomes a bottleneck, hindering effective intrusion detection at scale. \wajih{Again you need to add more details here about scalability issues. Having specific numbers for existing PIDS is important. Also cite those existing PIDS }
\end{itemize}

To address these challenges, we introduce \Sys, an innovative IDS that combines provenance graph representation learning with federated learning. This approach enables efficient, accurate detection of APT attacks in a decentralized manner, preserving user privacy. We have developed various novel techniques to overcome the inherent challenges of applying federated learning to PIDS, such as client data heterogeneity and imbalance. Specifically, we introduce a personalized \gnnshort framework tailored to the entity level and a dual-server architecture designed to safeguard user log privacy.


\wajih{Add one paragraph here explaining that why is it challenging to directly apply FL to PIDS.}

\wajih{Add one paragraph how you solve that above mentioned challenge. This will highlight the novelty of your system.}

\wajih{Add one paragraph related to evaluation results.}


\wajih{Add list of main contributions}