 \section{Evaluation}
 \label{sec:eval}

%  \wajih{Be consistent with datasets in each research question. Sometimes you use OpTC and sometimes you use E3 and you don't even explain why you excluded the other datasets. Rule of thumb is use all the datasets for the experiments unless you have reason or justification why you excluded certain datasets.}

 %\mati{Evaluation section has some writing inconsistencies. I will make a pass over it later today to fix it.}

%  \wajih{the word DISTDET is not consistent in the paper.}

 %\wajih{Use macros for common words in this whole section.}

 %\wajih{Be consistent with the terminology that you used in the abstract/intro.}

 %\wajih{Captions need to be lowercase}



%We evaluate \Sys using the open-source datasets \darpa E3 and \optc, which comprise system audit logs that simulate enterprise environments. These logs are collected from both Windows and Linux operating systems.



%\wajih{We also need to handle as ORTHRUS in the related work or evaluation section. I would love to hear creative way we can handle that.}

%\wajih{Handle this comment in the evaluation section: "I'm quite concerned about the evaluation results in Table 2. It doesn't look right to me that the number of malicious nodes in DARPA TC datasets can be sometimes around 20\%!"}

%\wajih{ We claim that the categorization-based ensemble improves convergence. But there is no empirical convergence analysis in the evaluation section e.g., training loss curves or number of FL rounds, etc. }

%\wajih{Add in the evaluation section how many clients we have in each dataset. "I can't find the number of FL clients in your text" }

Our evaluation experiments are conducted on a machine running Ubuntu 18.04.6 LTS, equipped with a 10-core Intel CPU, NVIDIA RTX 2080 GPU, and 120 GB of memory. In our experiments, we set the federated learning rounds and the number of categorized \gnnshort to 10 per host. Each model is trained for 20 epochs per round. We use regularization and dropout layers in our models to avoid overfitting. To evaluate \Sys, we address the following research questions: %\wajih{I have moved things around so align the RQs with that.}

%\wajih{RQ6 is not in this section. so please remove it. }

%\wajih{Add section references like I added for RQ1}

% \Fix{There are eight subsection in the evaluation while we have six research question. Make them consistent. This can confuse reader. Also, if you moved something to Appendix, refer them below. }

% \wajih{These are just too many research question. You need to combine similar RQs into one. If you think something is an ablation study then move it to the ablation study section. }
\begin{itemize}[leftmargin=*,itemsep=0.1em, parsep=0em, topsep=0em]
  \item \textbf{RQ1.} How does \Sys compare to vanilla privacy-preserving \pids in terms of detection performance? (Section~\ref{sub:detect:perf:vanilla})
  \item \textbf{RQ2.} How does \Sys compare to existing systems in terms of detection performance? (Section~\ref{sub:detect:perf})
  \item \textbf{RQ3.} How scalable is \Sys in an enterprise-level setting with multiple host machines? (Section~\ref{cost_metric})
  \item \textbf{RQ4.} How does \Sys compare to existing FL solutions in addressing data heterogeneity and non-IID challenges? (Section~\ref{sec:fedalternatives})
  \item \textbf{RQ5.} How robust is \Sys against various adversarial attacks?
  %\item \textbf{RQ5.} How effective is the Word2vec harmonization scheme for handling feature heterogeneity? (Appendix~\ref{sub:word2vec:harmonization:efficacy})
  %\item \textbf{RQ6.} How effective is the categorization-based \gnnshort ensemble for handling data heterogeneity? (Appendix~\ref{sub:categorized:learning:efficacy})
  
  % \item \textbf{RQ6.} How robust is \Sys against adversarial mimicry attacks?
  \item \textbf{RQ6.} What is the resource consumption of various components of \Sys and its end-to-end processing time on a client machine? (Appendix~\ref{sec:resource_consumption})
  \item \textbf{RQ7.} What does the ablation study reveal about \Sys's effectiveness across key factors like \gnnshort submodels, averaging rounds, federated averaging rounds, \wordvec harmonization and categorization-based \gnnshort ensemble? (Appendix~\ref{app:ablation})
  
  \end{itemize}  


\PP{Implementation} \Sys is developed in Python with around 5500 lines of code. It leverages the PyTorch and Torch Geometric libraries to implement the federated provenance graph learning framework. The graph learning framework uses the GraphSAGE~\cite{hamilton2017inductive} family of \gnnshort. Our architecture consists of two graph convolution layers with a Tanh activation function in between. The last layer uses a softmax function to output class probabilities for the nodes. For implementing the \wordvec model, we employ the Gensim library. Secure communication between clients and the utility server is ensured through Python's Cryptography module. The federated averaging, semantic vector harmonization, and entity categorization modules are implemented as individual Python functions on the central and utility servers.

% \PP{Datasets} We have utilized the \darpa E3~\cite{error3}, E5~\cite{bug5}, and \optc~\cite{darpaoptc} datasets for our evaluation. The E3 and E5 datasets consist of several adversarial engagements that simulate real-world APTs on enterprise networks. In these exercises, the red team aims to exploit vulnerabilities in the enterprise's services while hiding their attacks behind benign system activities. The logs captured from these exercises are documented under various scenario names, including Cadets, Trace, Theia and ClearScope. The \optc dataset, another open-source resource from \darpa, encompasses a comprehensive collection of audit logs from an enterprise environment with 1,000 hosts. This dataset includes six days of benign system logs, serving as training data for our system to learn normal behavior patterns. Subsequently, attack logs span three days of system activities, featuring red team tactics such as initial compromises, privilege escalations, malicious software installations, and data exfiltration. Each of these datasets is accompanied by ground truth documents that facilitate the distinction between benign and malicious events. For our evaluation, we employ attack labels from existing systems, such as \threatrace, \kairos and \flash for our evaluation. We evaluate our system using the same subset of datasets as existing open-source systems such as \kairos and \flash, and we utilize the attack node labels provided by them to ensure fairness.

\begin{table*}[!t]
  \centering
  \scriptsize
  \caption{Comparison of \Sys against vanilla privacy-preserving PIDS as baseline.}
  \setlength{\tabcolsep}{4pt} % slightly narrower columns
  \renewcommand{\arraystretch}{1} % optional: improves vertical spacing
  \begin{tabular*}{\textwidth}{@{\extracolsep{\fill}}lcccccccccccccc}
    \toprule
    \multirow{2}{*}{\textbf{Dataset}} &
    \multicolumn{7}{c}{\textbf{Baseline}} &
    \multicolumn{7}{c}{\textbf{\Sys}} \\
    \cmidrule(lr){2-8} \cmidrule(lr){9-15}
    & Precision & Recall & F1-score & TP & FP & FN & TN
    & Precision & Recall & F1-score & TP & FP & FN & TN \\
    \midrule
    E3-CADETS       & 0.64 & 0.99 & 0.78 & 12846 & 7243 & 4 & 699725
                    & \TCP & \TCR & \TCF & \TCTP & \TCFP & \TCFN & \TCTN \\
    E3-TRACE        & 0.85 & 0.99 & 0.92 & 67357 & 11390 & 20 & 2404623
                    & \TTP & \TTR & \TTF & \TTTP & \TTFP & \TTFN & \TTTN \\
    E3-THEIA        & 0.69 & 0.99 & 0.81 & 25314 & 11337 & 38 & 3493999
                    & \TTHP & \TTHR & \TTHF & \TTHTP & \TTHFP & \TTHFN & \TTHTN \\
    OpTC            & 0.36 & 0.99 & 0.53 & 649 & 1142 & 1 & 1286213
                    & \TOP & \TOR & \TOF & \TOTP & \TOFP & \TOFN & \TOTN \\
    E5-CADETS       & 0.76 & 0.99 & 0.86 & 40098 & 12356 & 10 & 1258638
                    & \ETCP & \ETCR & \ETCF & \ETCTP & \ETCFP & \ETCFN & \ETCTN \\
    E5-THEIA        & 0.73 & 0.99 & 0.84 & 54810 & 20767 & 270 & 1250910
                    & \ETTHP & \ETTHR & \ETTHF & \ETTHTP & \ETTHFP & \ETTHFN & \ETTHTN \\
    E5-ClearScope   & 0.79 & 0.97 & 0.88 & 13670 & 3491 & 397 & 79857
                    & \ETClP & \ETClR & \ETClF & \ETClTP & \ETClFP & \ETClFN & \ETClTN \\
    \bottomrule
  \end{tabular*}
  \label{summary:benchmarks:vanilla}
\end{table*}


\PP{Datasets} We have utilized the \darpa E3~\cite{error3}, E5~\cite{bug5}, and \optc~\cite{darpaoptc} datasets for our evaluation. These datasets contain real-world APT attacks observed in enterprise networks. The attack traces they include are stealthy, span several days, and mirror the characteristics of actual APTs. Consequently, achieving strong detection accuracy on these datasets indicates that our system can deliver comparable performance in real-world deployments. Furthermore, these datasets incorporate logs of various sizes from Linux, FreeBSD, Android, and Windows operating systems. Our system’s robust detection accuracy across all datasets demonstrates its effective generalization to heterogeneous platforms with differing log sizes, reaching performance levels on par with state-of-the-art centralized \pids. Notably, the datasets capture APT attacks of varying stealthiness, with the proportion of malicious nodes ranging from 0.05\% in \optc to 2\% in E3, and E5. The low infiltration rate in \optc underscores the system’s capacity for detecting highly stealthy adversaries, whereas the more widespread attacks in E3 and E5 highlight the resilience of our approach when confronted with a higher density of malicious nodes. Collectively, these datasets serve as a strong benchmark to evaluate the scalability and adaptability of our system. Each \darpa dataset is accompanied by ground truth documents that aid in distinguishing benign events from malicious ones. For this evaluation, we employ attack labels from existing systems such as \threatrace, \kairos, and \flash. Further details regarding the datasets appear in Appendix~\ref{sec:dataset:description}. ATLASv2~\cite{riddle2023atlasv2} is another recent dataset containing APT attack traces, but we did not evaluate it because it only includes data from two hosts, making it unrepresentative of a typical federated learning scenario.



\PP{Detectors for comparison} To benchmark our system, we compare against state-of-the-art \pids. \threatrace~\cite{wang2022threatrace} is a node-level system using graph representation learning to detect anomalous nodes in provenance graphs. MAGIC~\cite{jia2023magic} applies masked graph representation learning to identify threats. \flash~\cite{flash2024}, another node-level system, leverages semantic feature vectors and an embedding recycling database for enhanced detection and efficiency. As shown in Table~\ref{summary:benchmarks:large}, \flash surpasses  and thus serves as our primary baseline. \Fix{We also include \orthrus~\cite{jiang2025orthrus} and \kairos~\cite{cheng2023kairos}, which use temporal graph networks to capture system behavior over time.} We exclude Streamspot~\cite{streamspot}, Unicorn~\cite{han2020unicorn}, and \threatrace as they are outperformed by \flash and \kairos. While \disdet~\cite{dong2023distdet}, Prographer~\cite{yangprographer}, and Shadewatcher~\cite{shadewatcher} are notable, we exclude them as \disdet and Prographer are closed-source, and Shadewatcher relies on proprietary components, limiting reproducibility. Moreover, \flash and \orthrus have already demonstrated superior performance over Prographer~\cite{yangprographer} and Shadewatcher~\cite{shadewatcher}. We provide more details on why \disdet is unsuitable for comparison in Section~\ref{s:relwk}.

\Fix{It is important to note that, similar to existing works (e.g., \kairos, Shadewatcher, and Prographer), \Sys considers only three node types in provenance graphs: \emph{processes}, \emph{files}, and \emph{sockets}. However, in the E3 dataset, \flash has also been evaluated using additional node types. Therefore, we executed \flash using these three node types to report the results in Table~\ref{summary:benchmarks:large}.}

\subsection{RQ1: Detection Performance Against a Vanilla Privacy-Preserving PIDS}
\label{sub:detect:perf:vanilla}

%\wajih{Add a line about Vanilla Privacy-Preserving PIDS and how it was created by using basic FL with Flash. and say that Besides Flash we tried other SOTA PIDS and we got similar results for Vanilla Privacy-Preserving PIDS so we only include results for Flash.}

We conducted experiments to analyze the detection performance of \Sys in comparison to a vanilla privacy-preserving \pids, constructed by naively applying FL to an existing centralized \pids such as \flash. To simulate this setup, we operated \flash in a decentralized manner: each client locally trained \wordvec to encode semantic features and then trained their own \gnnshort models. These models were subsequently aggregated into a global model using the standard federated averaging algorithm. The results are shown in Table~\ref{summary:benchmarks:vanilla}. \Sys consistently outperforms the vanilla FL \flash across all detection metrics. This performance gain stems from \Sys's use of \wordvec harmonization and categorization-based ensemble learning, which effectively handle data heterogeneity. In contrast, the vanilla FL \flash lacks any mechanism to address this heterogeneity, resulting in degraded detection performance.



 \subsection{RQ2: Detection Performance Against SOTA PIDS}
 \label{sub:detect:perf}


\documentclass[conference]{./sty/IEEEtran-3}
%% To suppress warnings
\usepackage{silence}
\WarningFilter*{caption}{Unsupported document class}
%%%%%%%%%%%%%%%%%%%%%%% Packages %%%%%%%%%%%%%%%%%%%%%%%%%%%%
%% For urls
\usepackage[hyphens]{url}
\usepackage[breaklinks,colorlinks]{hyperref}
\usepackage[usenames,dvipsnames]{xcolor}
\usepackage[square,comma,numbers,sort&compress]{natbib}
%%
\usepackage{fancyhdr,lastpage}
\usepackage{epsfig}
\usepackage{balance} % balance bibliography
\usepackage{color}
\usepackage{comment}
\usepackage{enumitem}
\usepackage{epic}
\usepackage{epsf}
\usepackage{listings}
\usepackage{makecell}% http://ctan.org/pkg/makecell
\usepackage{multirow}
\usepackage{textcomp}
\usepackage{xspace}    % sticks a sane space after a command
\usepackage{latexsym}
\usepackage[labelfont=bf,font=small,skip=5pt]{caption}
\usepackage{afterpage}
\usepackage{amstext}   % provides \text{} command in math mode
\usepackage{amsmath}
\usepackage{graphicx}
\usepackage{tabularx}
\usepackage[caption=false]{subfig}
\usepackage{longtable}
\usepackage{rotating}
\usepackage[algoruled,linesnumbered]{algorithm2e}
% for math macro and numbers
\usepackage{fp}
\usepackage{siunitx}
\usepackage{xstring}
\usepackage{multirow}

\usepackage{tabularx}
\usepackage{ragged2e}
\usepackage{amssymb}
\usepackage{booktabs}

\usepackage{algorithmicx}
\usepackage{algpseudocode}
%%%%%%%%%%%%%%%%%%%%%%% END of Packages %%%%%%%%%%%%%%%%%%%%%%%%%%%%

% use \num{123456} -> 123,456
\sisetup{group-separator={,},group-minimum-digits={3},output-decimal-marker={.}}
%
\pagestyle{fancy}
\fancyhf{}
\renewcommand{\headrulewidth}{0pt}
\cfoot{\thepage}
% Define Colors
\definecolor{bblue}{HTML}{4F81BD}
\definecolor{rred}{HTML}{C0504D}
\definecolor{ggreen}{HTML}{9BBB59}
\definecolor{ppurple}{HTML}{9F4C7C}
\definecolor{darkgray}{rgb}{0.66, 0.66, 0.66}
\definecolor{gray}{RGB}{136,136,136}
\definecolor{dkgreen}{rgb}{0,0.6,0}
\definecolor{gray}{rgb}{0.5,0.5,0.5}
\definecolor{mauve}{rgb}{0.58,0,0.82}
\definecolor{comment-red}{rgb}{0.8,0,0}

\hypersetup{citecolor=ppurple,linkcolor=ppurple,urlcolor=black}
\urlstyle{sf}

% Macros
\newcommand\inlineeqno{\stepcounter{equation}\ (\theequation)}
\newcommand\mycommfont[1]{\footnotesize\ttfamily{\scriptsize \texttt{\textcolor{dkgreen}{#1}}}}
\SetCommentSty{mycommfont}
\renewcommand{\rothead}[2][60]{\makebox[9mm][c]{\rotatebox{#1}{\makecell[c]{#2}}}}
\makeatletter
\newcommand\footnoteref[1]{\protected@xdef\@thefnmark{\ref{#1}}\@footnotemark}
\makeatother
%%% Meta MA
\newcommand{\DefMacro}[2]{\expandafter\newcommand\csname rmk-#1\endcsname{#2}}
\newcommand{\UseMacro}[1]{\csname rmk-#1\endcsname}
\newcommand{\Caption}[1]{\caption{#1}}
\newcommand{\CodeIn}[1]{{\small \texttt{#1}}}
%%\newcommand{\Comment}[1]{}
\newcommand{\Space}[1]{}
\newcommand{\NA}{-}
\newcommand{\Fix}[1]{\textcolor{red}{#1}}
\newcommand{\Bullet}{$\star$}

\newcommand{\TODO}[1]{\pgwrapper{TODO}{#1}}
\makeatletter
\newcommand{\labitem}[2]{%
\def\@itemlabel{\textbf{#1}}
\item
\def\@currentlabel{#1}\label{#2}}

\newcommand{\RN}[1]{%
  \textup{\uppercase\expandafter{\romannumeral#1}}%
}
\SetKwProg{Fn}{Function}{}{}
% MATH
\newcommand{\shl}{\ \cc{<}\cc{<}\ }
\newcommand{\shr}{\ \cc{>}\cc{>}\ }
\newcommand{\x}{$\times$\xspace}
\newcommand{\PP}[1]{
\vspace{2px}
\noindent{\bf \IfEndWith{#1}{.}{#1}{#1.}}
}
\renewcommand{\paragraph}[1]{\smallskip\noindent\emph{#1}\quad}
\newcommand{\Norothead}[2][0]{\makebox[9mm][c]{\rotatebox{#1}{\makecell[c]{#2}}}}
\newcommand{\V}{\checkmark}
\newcommand{\X}{{\footnotesize $\times$}\xspace}
\renewcommand{\O}{\phantom{0}}
\newcommand{\etal}{\textit{et al}.\xspace}
\newcommand{\ie}{\textit{i}.\textit{e}.}
\newcommand{\eg}{\textit{e}.\textit{g}.}
\newcommand{\URL}{\url}
% Choose the values of the following 2 length parameters to suit your needs:
\newlength\tbspace
\setlength\tbspace{3mm}
\newcolumntype{C}{c<{\hspace{\tbspace}}}
%% \newlength\lengtha \setlength\lengtha{3mm}
%% \newlength\lengthb \setlength\lengthb{8mm}
%% \newcolumntype{C}{@{\extracolsep{8mm}}c@{\extracolsep{0.5em}}}%

%%%%%%%%%%%%%%%% Custom Macros For Paper  %%%%%%%%%%%%%%%%
\newcommand{\wajih}[1]{\textcolor{red}{Wajih: #1}}
\newcommand{\Sys}{\mbox{\textsc{TrustWatch}}\xspace}
% \newcommand{\sys}{\mbox{\textsc{XXX}}\xspace}
\newcommand{\Cost}{XXX\%\xspace}
\newcommand{\gnn}{Graph Neural Network\xspace}
\newcommand{\gnnshort}{GNN\xspace}
\newcommand{\grl}{Graph Representation Learning\xspace}
\newcommand{\grlshort}{GRL\xspace}
\newcommand{\provenance}{provenance\xspace}
\newcommand{\threatrace}{ThreaTrace\xspace}
\newcommand{\fscore}{F-Score\xspace}
\newcommand{\optc}{OpTC\xspace}
\newcommand{\unicorn}{Unicorn\xspace}
\newcommand{\streamspot}{StreamSpot\xspace}
\newcommand{\xgb}{XGBoost\xspace}
\newcommand{\lines}{1000\xspace}
\newcommand{\wordvec}{Word2Vec\xspace}
\newcommand{\graphsage}{GraphSage\xspace}

\newcommand{\shadewatcher}{ShadeWatcher\xspace}
\newcommand{\prographer}{ProGrapher\xspace}

\newcommand{\bigEI}[1]{$\mathcal{E}_{#1}$\xspace}
\newcommand{\IDS}{intrusion detection system\xspace}
\newcommand{\darpa}{DAPRA\xspace}
\newcommand{\provdetector}{ProvDetector\xspace}
\newcommand{\downstream}{lightweight\xspace}
\newcommand{\provids}{provenance-based IDSes\xspace}
\newcommand{\Provids}{Provenance-based IDSes\xspace}

%%====================================================================================================

%% Cadets-Threatrace
\newcommand{\TCP}{0.90\xspace}
\newcommand{\TCR}{0.99\xspace}
\newcommand{\TCF}{0.95\xspace}
\newcommand{\TCTP}{12848\xspace}
\newcommand{\TCFP}{1361\xspace}
\newcommand{\TCT}{334\xspace}

%% Fivedirections-Threatrace
\newcommand{\TFP}{0.67\xspace}
\newcommand{\TFR}{0.92\xspace}
\newcommand{\TFF}{0.78\xspace}
\newcommand{\TFTP}{389\xspace}
\newcommand{\TFFP}{188\xspace}
\newcommand{\TFT}{706\xspace}

%% Theia-Threatrace
\newcommand{\TTHP}{0.87\xspace}
\newcommand{\TTHR}{0.99\xspace}
\newcommand{\TTHF}{0.93\xspace}
\newcommand{\TTHTP}{25297\xspace}
\newcommand{\TTHFP}{3765\xspace}
\newcommand{\TTHT}{1288\xspace}

%% Trace-Threatrace
\newcommand{\TTP}{0.72\xspace}
\newcommand{\TTR}{0.99\xspace}
\newcommand{\TTF}{0.83\xspace}
\newcommand{\TTTP}{67382\xspace}
\newcommand{\TTFP}{26774\xspace}
\newcommand{\TTT}{912\xspace}

%% Optc Threatrace
\newcommand{\TOAP}{0.85\xspace}
\newcommand{\TOAR}{0.86\xspace}
\newcommand{\TOAF}{0.85\xspace}
\newcommand{\TOATP}{566\xspace}
\newcommand{\TOAFP}{99\xspace}
\newcommand{\TOAFN}{84\xspace}

%%====================================================================================================

%% Cadets-TrustWatch
\newcommand{\FCP}{0.92\xspace}
\newcommand{\FCR}{0.99\xspace}
\newcommand{\FCF}{0.95\xspace}
\newcommand{\FCTP}{12851\xspace}
\newcommand{\FCFP}{1062\xspace}

%% Fivedirections-TrustWatch
\newcommand{\FFP}{0.36\xspace}
\newcommand{\FFR}{0.99 \xspace}
\newcommand{\FFF}{0.53\xspace}
\newcommand{\FFTP}{420\xspace}
\newcommand{\FFFP}{750\xspace}

%% Theia-TrustWatch
\newcommand{\FTHP}{0.34\xspace}
\newcommand{\FTHR}{0.99\xspace}
\newcommand{\FTHF}{0.50\xspace}
\newcommand{\FTHTP}{25317\xspace}
\newcommand{\FTHFP}{52192\xspace}

%% Trace-TrustWatch
\newcommand{\FTP}{0.49\xspace}
\newcommand{\FTR}{1.0\xspace}
\newcommand{\FTF}{0.66\xspace}
\newcommand{\FTTP}{67383\xspace}
\newcommand{\FTFP}{68947\xspace}

%% Optc-TrustWatch
\newcommand{\FOAP}{0.86\xspace}
\newcommand{\FOAR}{0.95\xspace}
\newcommand{\FOAF}{0.90\xspace}
\newcommand{\FOATP}{620\xspace}
\newcommand{\FOAFP}{100\xspace}






\begin{document}

\title{Private Yet Accurate: A Decentralized Approach to System Intrusion Detection}

\maketitle

\begin{abstract}

    Enterprises increasingly rely on Intrusion Detection Systems (IDS) to detect malicious threats through log analysis. However, centralizing these logs, which often contain sensitive data (e.g., URLs), raises privacy concerns. In this paper, we propose \Sys that leverages a novel adaptation of Federated Learning (FL) and graph representation learning to enable local model training on \logs at client machines without sending raw logs to a central server, thus promoting decentralization and privacy in \ids. FL presents challenges in \ids due to varying data distribution and heterogeneity among clients, which impacts the generalization of the global model. To mitigate this, we introduce a novel ensemble learning approach and a process entity categorization scheme, where each submodel specializes in distinct system activity patterns across clients, preventing conflation during model aggregation.  Semantically rich feature vectors are crucial for high detection accuracy; however, semantic encoders, such as \wordvec complicate private aggregation on a central server in the FL process. To resolve this, we developed a \wordvec harmonization framework within a multi-server architecture that securely aggregates semantic attributes. One server generates encryption keys for client-level log attributes, while another performs computations on encrypted data without revealing sensitive tokens. Extensive evaluations on DARPA E3, E5, and OpTC datasets show that our system achieves state-of-the-art detection accuracy while being highly scalable and privacy-preserving.
\end{abstract}

% Data provenance transforms \logs logs into detailed provenance graphs, offering a better understanding of host activities. When combined with Graph Neural Networks (GNNs) and FL, it becomes a powerful technique for differentiating benign from malicious behaviors.

% We have designed a multi-server architecture and a robust encryption scheme to preserve the privacy of important user log attributes. To address the challenges of non-IID data distributions, heterogeneous, and imbalanced clients, we implement a sophisticated GNN learning framework personalized at the system entity level. We also present a detailed analysis of our system's resilience against adversarial attacks. Extensive evaluation of our system on real-world datasets from DARPA demonstrates that it achieves state-of-the-art detection performance while being highly scalable and privacy-preserving.
% \begin{CCSXML}
<ccs2012>
   <concept>
       <concept_id>10002978.10002997</concept_id>
       <concept_desc>Security and privacy~Intrusion detection</concept_desc>
       <concept_significance>500</concept_significance>
       </concept>
   <concept>
       <concept_id>10002978.10002979.10002982.10011600</concept_id>
       <concept_desc>Security and privacy~Machine learning</concept_desc>
       <concept_significance>500</concept_significance>
       </concept>
   <concept>
       <concept_id>10002978.10003006.10003007</concept_id>
       <concept_desc>Security and privacy~Operating systems security</concept_desc>
       <concept_significance>500</concept_significance>
       </concept>
 </ccs2012>
\end{CCSXML}

\ccsdesc[500]{Security and privacy~Intrusion/anomaly detection}
\ccsdesc[500]{Security and privacy~Machine learning}
\ccsdesc[500]{Security and privacy~Operating systems security}
% \keywords{Data provenance; Federated learning; Intrusion detection}

\section{Introduction}
\label{s:intro}

Intrusion Detection Systems (IDSes) play an essential role in enterprise security strategies to counteract Advanced Persistent Threats (APTs). APTs are notably stealthy and persistent, exemplified by the significant system disruptions seen in attacks such as Solar Winds~\cite{solarwinds} and NotPetya~\cite{notpetya}. The effectiveness of IDSes hinges on their ability to accurately detect these threats, maintain low false positive rates, and operate with minimal resource consumption, ensuring system performance is not compromised.

Many organizations outsource their security operations to different MSSPs. A study~\cite{msspsurvey}  of more than 5,000 IT professionals found that nearly three in every four companies are turning to MSSPs The MSSP integrates their security tools with the client's systems to collect audit logs. This often involves configuring the client’s systems to send audit logs to the MSSP's security operations center (SOC) for analysis. The MSSP uses advanced analytical tools and techniques, such as machine learning algorithms , to analyze the log data. The goal is to identify suspicious patterns or anomalies that may indicate a security threat or intrusion. When a potential threat or intrusion is detected, the MSSP alerts the client 

%\wajih{Give the workflow of MSSP and cite that organizations outsource their security. If you find any numbers on how many comapnies use MSSP and outsource their security operations that would be great. If you find any realworld attack examples on MSSP where the data was leaked that would be great as well. }

Data Provenance techniques have been increasingly used in IDSes to detect system intrusions. By analyzing system (audit) logs and converting them into provenance graphs, these systems offer a comprehensive view of system execution. Provenance-based IDSes~\cite{streamspot,provdetector2020,wang2022threatrace,shadewatcher,yangprographer,han2020unicorn} have emerged as a potent solution, utilizing the rich context within audit logs to improve detection capabilities. Despite their promise, these systems face significant challenges in the complex landscape of enterprise security:

\begin{itemize} [leftmargin=*]
    \item[--] \textbf{Lack of Privacy Preservation:} Traditional provenance-based IDSes (PIDS)~\cite{flash2024,cheng2023kairos,wang2022threatrace} rely on a centralized infrastructure, necessitating client machines to transmit their log data to a central server. Flash~\cite{flash2024} and Kairos~\cite{cheng2023kairos} are some recent state-of-the-art PIDS that operate in a centralized manner, assuming the audit logs from different machines will be present in a centralized location where these systems are operational. This approach risks user privacy as audit logs encompass detailed records of system execution, including sensitive information and activity patterns.
    
    \item[--] \textbf{Excessive Network and Disk Overhead:} Audit logs from modern systems can accumulate to gigabytes per day, resulting in significant network costs for both users and organizations. Additionally, the need for extensive disk storage to house these logs adds complexity. Current PIDS models fail to address these challenges, undermining their practical applicability. Our analysis of Flash and Kairos, utilizing the \optc dataset as detailed in Section~\ref{sec:eval}, illustrates this point. For an organization similar to the one represented in the \optc dataset, with a comparable number of hosts, the total volume of logs transmitted daily would reach 1000 GB. This volume of data could lead to substantial network and storage expenses daily. Moreover, it poses a significant challenge for users with limited network bandwidth to upload this volume of data efficiently.
    %\wajih{Could you give specific numbers about network overhead and disk overhead that would be great.} \wajih{Again cite existing PIDS with names that would suffer from this issue.}
    
    \item[--] \textbf{Scalability Challenges:} The centralized approach used by current Provenance-based Intrusion Detection Systems (PIDS) leads to scalability challenges as the number of hosts in an organization increases. This centralization acts as a bottleneck, impeding efficient intrusion detection on a larger scale. Our evaluations of Flash and Kairos reveal that these systems would encounter log congestion in organizations comparable to the size represented by the \optc dataset. As detailed in Section~\ref{sec:eval}, Flash would require 27.7 hours, and Kairos 56.6 hours, to process a single day's worth of logs from the \optc dataset. %\wajih{Again you need to add more details here about scalability issues. Having specific numbers for existing PIDS is important. Also cite those existing PIDS }
\end{itemize}

To address these challenges, we introduce \Sys, an innovative IDS that combines provenance graph representation learning with federated learning. This approach enables efficient, accurate detection of APT attacks in a decentralized manner, preserving user privacy. We have developed various novel techniques to overcome the inherent challenges of applying federated learning to PIDS, such as client data heterogeneity and imbalance. Specifically, we introduce a personalized \gnnshort framework tailored to the entity level and a dual-server architecture designed to safeguard user log privacy.


\wajih{Add one paragraph here explaining that why is it challenging to directly apply FL to PIDS.}

\wajih{Add one paragraph how you solve that above mentioned challenge. This will highlight the novelty of your system.}

\wajih{make sure to say two or three lines about adversarial attacks and say that more details are present in the discussion section.}

\wajih{Add one paragraph related to evaluation results.}


\wajih{Add list of main contributions}


\PP{Open Source Commitment} To advance threat detection research and ensure reproducibility, we plan to release the source code of \Sys publicly upon the publication of this paper.
\section{Related work}
\label{s:relwk}

% \wajih{Check NDSS 2025 and USENIX 2024 and IEEE S\&P 2024 to see if there is a paper that we have not cited in the related. If there is provenance related paper or IDS we need to cite it in the Related Work.}


% \wajih{Google scholar search FL+IDS and see if there are some new papers we can cite in the related work.}


%\wajih{ Make sure to cite this paper: https://www.usenix.org/system/files/sec23winter-prepub-490-jia.pdf}

\PP{ML-based IDS} ML techniques are widely used in threat detection. ProvDetector~\cite{provdetector2020} applies Doc2Vec~\cite{le2014distributed} to provenance graphs; Attack2Vec~\cite{shen2019attack2vec} uses temporally aware embeddings. DeepAid~\cite{deepaid} classifies anomalous traffic, ProGrapher~\cite{yangprographer} combines Graph2Vec~\cite{narayanan2017graph2vec} and TextRCNN~\cite{lai2015recurrent}, and StreamSpot~\cite{streamspot} clusters graph features. Other systems~\cite{aljawarneh2018anomaly, maseer2021benchmarking, gyanchandani2012taxonomy, atlas} explore varied embeddings; some focus on malware~\cite{zolkipli2011approach, chakkaravarthy2019survey, isohara2011kernel}. DeepLog~\cite{deeplog2017} uses RNNs on logs, SIGL~\cite{sigl} detects software installs, Euler~\cite{king2022euler} combines GNNs and RNNs, and MAGIC~\cite{jia2023magic} uses masked graph learning. \Sys is the first to combine federated learning with provenance-based IDS, addressing privacy, scalability, and heterogeneity.

DistDet~\cite{dong2023distdet} detects APTs using Hierarchical System Event Trees (HSTs) to summarize local system activity and reduce network overhead. However, this summarization comes at the cost of expressiveness and privacy. HSTs must be transmitted in plaintext to a central server, exposing sensitive execution traces without any formal privacy guarantees. More critically, HSTs encode flat, linear event sequences and lack the structural richness of provenance graphs, limiting their ability to model complex causal and multi-entity interactions. Anomalies are detected by identifying sequences absent from a reference benign HST, a simplistic approach that fails to generalize in dynamic environments and results in frequent false positives.


\PP{Rule-based \pids} Rule-based \pids rely on predefined rules for detecting malicious activities. Examples include Holmes~\cite{holmes2019}, Rapsheet~\cite{rapsheet2020}, Poirot~\cite{poirot2019}, and CAPTAIN~\cite{wang2024incorporating} which leverage insights from APTs to construct rule bases, achieving fewer false positives compared to ML-based systems. However, these systems face significant limitations: they cannot detect threats with novel attack signatures and require skilled security professionals to design and update the rule sets. Additionally, none of these systems ensure privacy preservation during their operation.

\PP{Federating Learning in Threat Detection} Few IDS employ federated learning, with most research centered on Network Intrusion Detection. Examples include \cite{man2021intelligent}, proposing FL for IoT threat detection, and \cite{friha20232df}, which introduces a differentially private system for industrial IoT. \cite{li2023efficient} presents an efficient network intrusion detection framework, while \cite{guo2023new} addresses non-IID data issues in FL for intrusion detection. \Sys advances this area by being the first system to integrate FL with graph-based learning techniques for host-based threat detection using a categorization based \gnnshort ensemble framework and secure \wordvec harmonization. XFedGraph-Hunter \cite{son2023xfedgraph} employs FL and \gnnshort for network intrusion detection, while FedHE-Graph~\cite{mansour2024fedhe} introduces an FL-based intrusion-detection mechanism in a single-server setting. Unlike \Sys, however, their approach lacks semantic harmonization which is essential for leveraging semantic feature vectors and achieving detection performance on par with centralized \pids.

\PP{Cryptographic Techniques} Cryptographic techniques, such as Multi-Party Computation (MPC)~\cite{cramer2015secure} and Fully Homomorphic Encryption (FHE)~\cite{armknecht2015guide}, offer strong privacy guarantees. MPC allows multiple parties to compute a function over their inputs while keeping those inputs private, and FHE enables computations on encrypted data. However, these methods face significant challenges, including increased system complexity and scalability issues~\cite{du2001secure, gentry2009fully, asharov2013more}, particularly with large datasets~\cite{menezes2018handbook}. For instance, host intrusion detection systems often process terabytes of logs, making the computational overhead of MPC or FHE impractical for real-time threat detection~\cite{loggc}. In contrast, \Sys combines a scalable encryption technique with federated learning to ensure both scalability and privacy preservation. Section~\ref{sec:privacy} provides a detailed analysis of the privacy guarantees of this approach.

\PP{Privacy-preserving FL} PrivateFL~\cite{yang2023privatefl} tackles the heterogeneity caused by differential privacy (DP) in FL systems through personalized data transformations to protect model updates from inference attacks. Similarly, \cite{annamalai2023fp} applies FL and DP to detect browser fingerprinting, and \cite{dasu2022prov} introduces secure federated averaging using homomorphic encryption. Poseidon~\cite{sav2020poseidon} utilizes multiparty cryptography for privacy-preserving neural network training, while PpeFL~\cite{wang2023ppefl} adopts local DP for FL, addressing privacy and model performance issues. Unlike these methods, \Sys introduces a privacy-preserving multi-model server architecture that avoids DP-induced noise, ensuring robust performance while resisting inference attacks. We provide experimental results on the issue of applying DP for the PIDS domain in Appendix~\ref{app:dp}.

% \wajih{papers to cite:}
% https://www.usenix.org/system/files/usenixsecurity23-yang-yuchen.pdf


% https://www.ndss-symposium.org/ndss-paper/fp-fed-privacy-preserving-federated-detection-of-browser-fingerprinting/

% https://dl.acm.org/doi/10.1145/3560830.3563729

% https://www.ndss-symposium.org/ndss-paper/poseidon-privacy-preserving-federated-neural-network-learning/

% https://ieeexplore.ieee.org/document/10091486


% https://arxiv.org/abs/2310.20552

% https://dl.acm.org/doi/10.1145/3624017


% \wajih{You need to go through these papers, especially their threat model and evaluation section. Tell me what metrics are they using to prove that their FL is privacy-preserving. And you need to tell me why we can or cannot apply to our system.}

% \section{Background \& Motivation}
\label{sec:motivation}
% Contemporary host intrusion detection systems, exemplified by \unicorn~\cite{han2020unicorn}, \streamspot~\cite{streamspot}, \threatrace~\cite{wang2022threatrace} and \prographer~\cite{yangprographer}, heavily rely on system audit log data to identify malicious entities within a system. Employing advanced deep learning techniques such as \gnn (\gnnshort), these systems strive to enhance the accuracy of threat detection. However, the efficacy of these techniques is contingent upon vast amounts of data to train the underlying models, often reaching terabytes in size. Acquiring such extensive training data from a single user machine is impractical.

% Our investigation, involving the simulation of normal user workloads on test machines, reveals that the audit logs generated from these activities are relatively small in volume. Consequently, they prove insufficient for adequately training these large-scale machine learning models. To address this limitation, it becomes imperative to aggregate data from a diverse array of machines to a centralized storage system for comprehensive model training.

% Nevertheless, the consolidation of audit logs from various sources poses a significant challenge due to the inherent risk of privacy leakage. These logs contain vital information about the diverse activities carried out by different users, encompassing details about utilized applications, browsing history, and sensitive data like email content, phone numbers, as well as financial and medical information. These privacy concerns are underscored in a report by Datadog~\cite{datadog}, a prominent provider of system monitoring services. Consequently, utilizing existing systems for detecting system threats introduces a potential compromise of user privacy.

% Moreover, the centralized aggregation of all data elevates the risk of data leakage and compromises the efficiency of these systems in terms of both memory utilization and runtime efficiency. As a result, there is a pressing need for innovative solutions that balance the imperative of robust threat detection with the paramount importance of safeguarding user privacy and system efficiency.

% Federated learning (FL) is an establish technique for privacy preserving machine learning. In federated learning the individual client data does not leave the system. Instead each client machine trains a local machine learning model on its local data and then these clients sends the trained model to a central server where the federated averaging is utilized to combine the information from these models to get a unified global model. However our experimentation with existing systems reveal that applying federated learning to them yields poor results because these systems are not designed to work in this fashion.

\wajih{make a macro for Provenance-based IDS which is PIDS and use PIDS everywhere where you say intrusion detection. It saves space and also narrows the scope of our paper to PIDS. Also, make a table for the limitations section so that I can point to it in the introduction section. Use macros for words like word2vec, etc. because of letter capitalization. Also convert the words like Graph Neural Network into GNN to save space.}


In this section, we examine existing host intrusion detection systems and underscore their design flaws, particularly regarding the preservation of user log privacy. Flash~\cite{flash2024}, a node-level anomaly detection system, utilizes a \gnn to learn standard system behavior from provenance graphs. This system adopts an advanced featurization method, using a temporal ordering-aware word2vec model to capture both structural and semantic characteristics in audit logs. Moreover, Flash enhances inference speed via an \gnnshort embedding store. In contrast, \threatrace~\cite{wang2022threatrace}, another node-level detection system, also employs a \gnn for anomaly detection. However, it stands apart from Flash due to its scalability issues and reliance on basic features, which limits its effectiveness in fully leveraging the information contained in provenance graphs. Kairos~\cite{cheng2023kairos}, an anomaly-based detector, segments logs into time window queues and uses temporal graph neural networks for anomaly detection. Despite the progress made by existing intrusion detection systems, they fail to provide any privacy safeguards for sensitive information that may be contained in the audit logs they utilize. This deficiency can impede the deployment of these systems in domains where privacy is paramount. We expand upon these points in the following discussion: 

\PP{Data Privacy} The existing intrusion detection systems that have been discussed above predominantly rely on system audit log data to identify malicious entities within a system. These systems employ advanced deep learning techniques, such as \gnn, to achieve impressive detection results. However, a notable downside of these models is their substantial requirement for data to accurately learn benign system behavior. This volume of data is impracticable to generate from a single user's machine. Consequently, in an enterprise environment, it becomes essential to collect data from a multitude of machines in order to compile a sufficiently large dataset for training these detection systems. \wajih{We have some experiments for this as well to show why it is necessary to do get data from some machines for better models. Put numbers or figure related to those experiments here.}


However, this approach introduces a significant risk to user privacy. When a central entity analyzes these logs, it gains insights into the users' system activities. This can range from identifying the applications they use, to the websites they browse, and even potentially inferring their location from network IPs. These privacy concerns have been underscored in a report by Datadog~\cite{datadog}, a prominent provider of system monitoring services. Thus, while existing intrusion detection systems are effective in identifying system threats, they simultaneously pose a potential compromise to user privacy.

\PP{High Network and Disk Overhead} The current systems function under the premise of a centralized infrastructure designed for aggregating user logs to train models. In this setup, user machines periodically transmit their logs to a central server, which then consolidates them in a centralized database. However, the volume of these log data can reach gigabytes, resulting in significant network expenses for both users and the organization. Additionally, it demands substantial disk storage to accommodate all the centralized logs. Such extensive network costs disproportionately affect users with limited bandwidth, as they may be unable to contribute their data for model training. This exclusion hampers the model's ability to learn a comprehensive representation of benign behavior, potentially leading to a higher rate of false alarms for those not participating. Moreover, the necessity for extensive disk storage can restrict organizations from incorporating data from all users, which might introduce data bias and degrade the model's performance in practical scenarios. \wajih{For this point, we discuss that we will do some calculations that how much network overhead it will incur to send one day worth of OpTC dataset and Cadets to the central server for anomaly detection. Add those numbers here.}

\PP{Centralized Learning \& Scalability} The existing systems train their deep neural models using logs stored centrally. This approach significantly prolongs the training time of the models. The challenge escalates when frequent retraining is required to address the issue of concept drift\Fix{~\cite{}}. Additionally, these systems lack inherent mechanisms for parallelizing the training process, leading to an excessively lengthy sequential training period\Fix{~\cite{}}. Furthermore, managing all the data in a single location poses challenges in efficiently utilizing system resources, such as CPU and memory.

Moreover, existing systems, such as Flash, have implemented techniques to achieve efficient runtime performance. However, in an enterprise context, the centralized mode of operation faces scalability limitations. It can only accommodate a fixed number of hosts before the central server, running the intrusion detection system, becomes a bottleneck. In high-throughput environments, this leads to the detector suffering from log congestion.

\begin{table}[h!]
    \centering
    \scriptsize
      \caption{Limitations of Existing Systems.}
      \setlength{\tabcolsep}{4.8pt}
        \begin{tabular}{ | c | c | c | c | c |}
          \hline
               & \bf Privacy & \bf Network  & \bf Disk  & \bf Scalability \\
               & \bf  Preserving & \bf  Cost & \bf Cost &  \\
          \hline
          \Sys & YES                & LOW          & LOW       & HIGH        \\
          \hline
          FLASH      & NO                 & HIGH         & HIGH      & MEDIUM      \\
          \hline
          KAIROS     & NO                 & HIGH         & HIGH      & LOW         \\
          \hline
          THREATRACE & NO                 & HIGH         & HIGH      & LOW        \\
          \hline
        \end{tabular}
        \label{limitations}
    \end{table}

\subsection{How can federated learning help to solve above challenges?}


\subsection{Why is it challenging to apply FL to PIDS?}

Our analysis indicates that applying federated learning to existing systems would produce suboptimal results. This is primarily because these systems were not originally designed to function under a federated learning framework.

\PP{Randomized Esemble Models} \threatrace employs an ensemble of \gnnshort models to comprehensively learn the distribution of benign system activities. In this ensemble, each subsequent model is trained on entities that were misclassified by the preceding model. However, since these models are trained on random segments of the graph, applying the federated averaging algorithm~\cite{mcmahan2017communication} is problematic. Naive combination of these models in a federated context could result in a loss of critical information in the global model, due to their distinct training data subsets.

\PP{Feature Space Heterogeneity} Flash utilizes a temporally-aware Word2Vec model to encode the semantic attributes of entities within the provenance graph. In a federated learning environment, each client machine independently trains its Word2Vec model on its feature set. However, inherent randomness in the Word2Vec algorithm means that identical tokens may be encoded into different vectors by each client. Consequently, when federated averaging is applied to \gnn models that use these features, the performance is notably poor.

\PP{Temporal Misalignment} Kairos leverages temporal graph neural networks to understand the evolution of a system's provenance graph over time. However, the application of federated learning to capture these temporal dependencies faces significant challenges. The fragmentation of data across various clients results in a lack of temporal alignment, which is crucial for the effectiveness of federated averaging in these network types. This misalignment impedes the ability to effectively integrate the learnings from individual client models into a cohesive global model.

\subsection{Overview of \Sys}
The aforementioned challenges substantially hinder the integration of the federated learning paradigm into existing systems. We have developed \Sys to address these obstacles. By merging federated learning with graph representation learning, \Sys provides a privacy-preserving detection system. In this framework, each client trains a \gnnshort model on their local log data. Subsequently, these models engage in a federated averaging process at a central server. To address the issue of heterogeneous feature vectors across different clients, \Sys incorporates a novel vector harmonization module.


\section{Threat Model \& Assumptions}

Our threat model assumes that the central server operates with integrity, conducting the federated averaging process without malicious objectives. Nonetheless, we acknowledge the possibility of the server exhibiting curiosity, potentially seeking to deduce characteristics of client training data from individual model weights. We assume that there is no collusion between the central and utility servers. The integrity of client-side data collections is assumed, alongside the absence of malicious clients during the training phase to prevent model poisoning.

On individual clients, we presume the presence of covert attackers who can disguise their malicious activities within benign data, complicating the distinction between malicious and normal actions. Additionally, we consider that the attackers can use zero-day exploits. We assume that the attacker leaves behind discernible traces of their activities in the system's records. Similar to prior works in data provenance such as~\cite{nodoze2019,priotracker2018,mzx2016,bates2017transparent}, our model assumes the provenance collection system accurately captures comprehensive records of all system activities and modifications. Furthermore, we assert the continuous integrity of audit logs assured by the use of tamper-resistant storage solutions~\cite{paccagnella2020custos,hardlog}.

\section{\Sys Design}
\label{sec:methodology}

In this section, we outline the architecture of our system, \Sys, which comprises five principal modules. The first module, the \textit{Provenance Graph Constructor}, operates on each client machine, transforming audit logs into a provenance graph. The second module, \textit{Semantic Featurization}, focuses on encoding semantic attributes from audit logs into feature vectors, which aids in the training of the client-specific \gnnshort models. This process utilizes word2vec models, each individually trained on its respective client's logs.

The third module, \textit{Semantic Vectors Harmonization}, is tasked with merging these individual word2vec models into a unified global model. This is achieved using a trusted utility server and encryption techniques to safeguard sensitive model data. 

Following this, the \textit{Federated Graph Learning Module} undertakes the task of training \gnnshort models on each client machine, utilizing the harmonized semantic features. Once the individual client models are trained, they engage in a federated learning process facilitated by the central server. In this process, each client transmits its model to the server, where the federated averaging algorithm is applied to amalgamate these models into a singular, comprehensive global model. This consolidated model is then dispatched back to the clients for further application and analysis.

Finally, the \textit{Anomaly Detection Module} utilizes the aggregated global model for local anomaly detection on each client machine. Figure \ref{arch} illustrates the overarching architecture of \Sys, with a more comprehensive explanation provided in the following subsections:

\begin{figure}[t!]
  \centering
  \includegraphics[width=0.45\textwidth]{fig/arch.pdf}
  \caption{High Level Architecture of \Sys.}%\wajih{Please reduce the different number of colors you are using in this diagram. They are distracting. Also, it is not clear from this diagram what is the order of operation. Maybe you need to put numbers on the edges. Or you need to make it from left to right to show which thing will happen first and then second and so on.}}
  \vspace{-3ex}
  \label{fig:arch}
\end{figure}

\subsection{Provenance Graph Constructor} 
\Sys harnesses audit logs to create a system provenance graph. It operates on each client machine, utilizing their local system logs for graph construction. Key operating systems, such as Linux and Windows, are equipped with built-in mechanisms, namely the Linux Audit system and Windows Event Tracing, for log collection. These logs offer comprehensive insights into the interactions among different system entities, documenting activities ranging from process executions and file operations to network connections. Utilizing this information, \Sys assembles a graph where nodes represent system entities like processes, files, and sockets. The edges of this graph correspond to events, predominantly identified by syscalls, that occur between these entities. In addition, \Sys augments each node with detailed attributes, including process names, command lines, file names, and network IP addresses. This enrichment of node attributes significantly bolsters our system's proficiency in differentiating between nodes with analogous graph structures.

\subsection{Semantic Featurization}
\wajih{use word2vec macro.}
% This module processes the provenance graph generated from audit logs by transforming node attributes into feature vectors for the graph learning phase. Existing systems like Flash have demonstrated the effectiveness of using semantic node attributes to enhance detection performance. Building on this approach, we employ a word2vec language model to encode these attributes into vector format. Each client independently trains a word2vec model using their local system logs. However, before these models can be utilized to encode text attributes, they must be merged across client machines to form a unified model. This unification is crucial; without it, each client would produce differing feature vectors for identical inputs. Such variability would undermine the consistency of client-based \gnnshort models and diminish the effectiveness of the federated averaging technique.

This module processes the provenance graph generated from audit logs by transforming node attributes into feature vectors for the graph learning phase. Systems like Flash have demonstrated the effectiveness of using semantic node attributes to enhance detection performance. Building on this approach, we employ a word2vec language model to encode these attributes into vector format, \(\mathbf{v}\), where each attribute \(a\) is transformed into a vector \(v_a\). Each client independently trains a word2vec model using their local system logs. The transformation of an attribute \(a\) into a vector by the word2vec model can be represented as:

\[
v_a = \text{word2vec}(a)
\]

However, before these models can be utilized to encode text attributes, they must be merged across client machines to form a unified model. This unification is crucial; without it, each client would produce differing feature vectors, \(v_a^i\), for identical inputs, where \(i\) indicates the client. The variability in feature vectors, \(\{v_a^1, v_a^2, \ldots, v_a^N\}\), for the same attribute \(a\) across \(N\) clients, would undermine the consistency of client-based GNN models. To ensure uniformity, the feature vectors for overlapping attributes must be averaged across clients, forming a single unified vector for each attribute:

\[
\bar{v}_a = \frac{1}{N} \sum_{i=1}^{N} v_a^i
\]

Such averaging ensures consistency in the feature representation, enhancing the effectiveness of the federated averaging technique by maintaining uniformity in the input space for the GNN models across all clients.


\subsection{Semantic Vectors Harmonization}
This module integrates individual client word2vec models into a unified model for use across all clients. The word2vec model functions as a key-value store, with vocabulary tokens as keys, \(k\), and their corresponding vector representations as values, \(v_k\). To combine these models, we calculate the average vector of overlapping tokens from all client machines, creating a central model. The mathematical representation for averaging vectors of a token \(k\) across \(N\) clients is given by:

\[
\bar{v}_k = \frac{1}{N} \sum_{i=1}^{N} v_{k,i}
\]

where \(\bar{v}_k\) is the averaged vector for token \(k\), and \(v_{k,i}\) is the vector representation of token \(k\) from the \(i\)-th client model.

However, transferring tokens—containing sensitive data like process names, file names, and IP addresses—to a central server could breach user privacy. To mitigate this, we employ a trusted utility server. Initially, the central server distributes an encryption key, \(E\), and a decryption key, \(D\), pair to each client. Clients then encrypt their word2vec model tokens using the encryption key:

\[
E(v_{k}) = v_{k}^{'}
\]

and send them to the utility server. This server merges the encrypted models and dispatches the unified model back to the clients, who decrypt it using the decryption key:

\[
D(v_{k}^{'}) = v_{k}
\]

This procedure ensures that neither the central server nor the utility server can access the actual token information, assuming no collusion between the two servers. The process is explained in detailed in algorithm~\ref{alg:secure_integration_averaging_word2vec}

\begin{algorithm}[h]
  \scriptsize
  \DontPrintSemicolon
  \SetKwInOut{Input}{Inputs}
  \SetKwInOut{Output}{Output}
  \Input{Client Word2Vec models $\{M_1, M_2, \ldots, M_N\}$ encrypted with key $E$; Encryption key $E$; Decryption key $D$.}
  \Output{Encrypted unified Word2Vec model $U'$ sent to clients.}
  \BlankLine
  \tcc{Distribute encryption and decryption keys to each client.}
  \ForEach{client $C_i$}{
    Send $E$ and $D$ to $C_i$\\
  }
  \tcc{Clients encrypt their model tokens.}
  \ForEach{client model $M_i$}{
    $M_i' \leftarrow$ EncryptModelTokens($M_i$, $E$) \tcc*{Encrypt tokens using $E$.}
    Send $M_i'$ to Utility Server\\
  }
  \tcc{Utility server merges encrypted models.}
  $TokenVectors \leftarrow$ InitializeEmptyDictionary()\\
  $TokenCounts \leftarrow$ InitializeEmptyDictionary()\\
  \ForEach{encrypted model $M_i'$}{
    \ForEach{token $t$ in $M_i'$}{
      $Vector \leftarrow M_i'[t]$\\
      \eIf{$TokenVectors$.HasKey($t$)}{
        $TokenVectors[t] \leftarrow TokenVectors[t] + Vector$\\
        $TokenCounts[t] \leftarrow TokenCounts[t] + 1$\\
      }{
        $TokenVectors[t] \leftarrow Vector$\\
        $TokenCounts[t] \leftarrow 1$\\
      }
    }
  }
  \tcc{Average the vectors for overlapping tokens.}
  \ForEach{token $t$ in $TokenVectors$.Keys()}{
    $TokenVectors[t] \leftarrow TokenVectors[t] / TokenCounts[t]$\\
  }
  $U' \leftarrow$ NewModel($TokenVectors$, $EncryptedTokens$) \tcc*{Constructing a new unified model.}
  \ForEach{client $C_i$}{
    Send $U'$ to $C_i$\\
  }
  \BlankLine
  \Return{Harmonized model $U'$ has been dispatched to all clients.}\\
  \BlankLine
  \caption{Secure Integration and Averaging of Word2Vec Models}
  \label{alg:secure_integration_averaging_word2vec}
\end{algorithm}

\subsection{Federated Graph Learning}
The module performs graph representation learning in a federated manner. It includes a central server responsible for initializing a global \gnnshort model with random weights, which is then sent to all clients. These clients use their local provenance graphs and semantic feature vectors to train the \gnnshort model in an unsupervised way, following a training method similar to Flash. The \gnnshort model's objective is to classify each node entity into its corresponding type. After training their models, the clients send them back to the central server. The server applies the federated averaging algorithm to merge these models into a single global model. The server aggregates parameters from $N$ client models to update the global model as follows:
\[
\bar{w} = \frac{1}{N} \sum_{k=1}^{N} w_k
\]

where:
\begin{itemize}
    \item $\bar{w}$ is the aggregated global model parameter.
    \item $N$ is the number of client models.
    \item $w_k$ is the parameter from the $k$-th client model.
\end{itemize}

This formula averages the parameters of all client models, forming the global model update. The process is repeated for a set number of rounds, determined by a hyperparameter, and concludes when there is no further reduction in training loss. Algorithm~\ref{alg:federated_learning} explains this process in detail.

\subsection{Anomaly Detection}
\Sys employs an advanced methodology akin to systems like Flash and \threatrace, focusing on identifying irregular nodes through the comparison of their expected and observed types. This approach is grounded in a detailed analysis of both the surrounding structures and inherent properties of the nodes, with the aim to define normal pattern baselines for various node types. Typically, entities with malicious intentions display neighborhood configurations and characteristics deviating from these established norms. In operational phases, the detection of anomalies that diverge from the pre-established node distribution patterns often results in their misclassification. The emergence of nodes misclassified in the system's output is indicative of potential security issues. To regulate the frequency of alerts, we have implemented a threshold parameter, denoted as $T$. This parameter sets a limit on the likelihood of a classification being considered valid. A higher value of this parameter implies stronger confidence in the prediction, increasing the probability of identifying anomalies.

\wajih{Whrere is the part about local differential privacy? We should have something about that in design section.}


% \subsection{Provenance Graph Constructor}
% Our approach starts by converting system logs into provenance graphs through a three-step process. Initially, the system, \Sys, processes system logs like Windows Event Logs or Linux Audit Logs, which are composed of host event details including process activities, file interactions, and network engagements. \Sys works with batches of audit logs, utilizing a sliding window technique to create the provenance graph. This graph consists of two kinds of nodes: process nodes and object nodes. The object nodes represent various system entities such as files, network streams, modules, and other system components. The connections between these nodes are marked with labels indicating the event type, elucidating the cause-and-effect relationship among the connected nodes and the event's timestamp. Additionally, these nodes are equipped with attributes like process identifiers, command lines, file paths, IP addresses, port details, and module paths, offering additional insights and specifics.

% \subsection{Semantic Vectors Harmonization}
% \wajih{You need to add formalism in this subsection overall. If there is a Math related to Harmonization add that. Look into the Flash paper -- we gave so much internal details about the algorithms. }
% Our system employs the word2vec model to encode various semantic attributes into a vector space, which is pivotal in distinguishing normal system entities from anomalies. Traditional approaches utilize a centralized word2vec model for this encoding. However, in a federated learning context, complexities arise as each individual client must train its word2vec model for attribute encoding. Consequently, different clients might encode the same attributes into diverse vectors, leading to a non-Independent and Identically Distributed (Non-IID) problem. This variation hampers the convergence of the Graph Neural Network (GNN) model in federated learning scenarios. 

% To address this issue, we have developed a technique leveraging a utility server to synchronize disparate models across clients. This approach achieves uniformity in encoding the same attributes while preserving privacy, as clients are not required to share their attributes. The process initiates with the main server distributing an encryption and decryption key to each client. Clients then encrypt their word2vec model tokens, concealing their meanings. Subsequently, these encrypted models are sent to the utility server, which remains unaware of the encryption key used. The utility server then averages the vectors for the corresponding encrypted tokens, creating a unified and harmonized word2vec model. Finally, this model is returned to the clients, who utilize the decryption key to revert their encrypted tokens to their original attributes.

% \subsection{Federated Learning Module}
% \wajih{You need to add formalism in this subsection overall. If there is a Math related to FL add that. Look into the Flash paper -- we gave so much internal details about the algorithms. }
% Each client machine independently trains a \gnn model on a provenance graph built from its local log data, thereby preserving privacy. These individually trained models are then sent to the main server. Upon receiving them, the main server applies a federated averaging algorithm to integrate these models into a single, centralized model. This unified model is subsequently distributed back to each client. This cycle of training and unification is repeated over several rounds until the model reaches convergence. Finally, clients employ the fully trained model to perform anomaly detection.

% \subsection{Anomaly Detection}
% \wajih{Again missing scientific details about your approach.}
% \Sys utilizes an advanced approach similar to existing systems like Flash and \threatrace to pinpoint irregular nodes by assessing the disparity between their expected and observed types. This method is rooted in a thorough examination of the surrounding structures and intrinsic qualities of nodes, aiming to establish a baseline of normal patterns for various node classifications. It is observed that entities with malicious intent often manifest neighborhood configurations and characteristics that are inconsistent with these established norms. During operational phases, encountering such anomalies that stand apart from the pre-learned node distribution patterns leads to their erroneous classification. The presence of nodes incorrectly categorized in the output serves as an indicator of potential security concerns. We have introduced a regulatory mechanism in the form of a threshold parameter, $T$, to oversee the frequency of alerts. This parameter effectively caps the classification likelihood for a given prediction. A higher parameter value correlates with stronger conviction in the prediction, signaling a greater chance of uncovering anomalies.

% \input{impl}
 \section{Evaluation}
 \label{sec:eval}

%  \wajih{Be consistent with datasets in each research question. Sometimes you use OpTC and sometimes you use E3 and you don't even explain why you excluded the other datasets. Rule of thumb is use all the datasets for the experiments unless you have reason or justification why you excluded certain datasets.}

 %\mati{Evaluation section has some writing inconsistencies. I will make a pass over it later today to fix it.}

%  \wajih{the word DISTDET is not consistent in the paper.}

 %\wajih{Use macros for common words in this whole section.}

 %\wajih{Be consistent with the terminology that you used in the abstract/intro.}

 %\wajih{Captions need to be lowercase}



%We evaluate \Sys using the open-source datasets \darpa E3 and \optc, which comprise system audit logs that simulate enterprise environments. These logs are collected from both Windows and Linux operating systems.



%\wajih{We also need to handle as ORTHRUS in the related work or evaluation section. I would love to hear creative way we can handle that.}

%\wajih{Handle this comment in the evaluation section: "I'm quite concerned about the evaluation results in Table 2. It doesn't look right to me that the number of malicious nodes in DARPA TC datasets can be sometimes around 20\%!"}

%\wajih{ We claim that the categorization-based ensemble improves convergence. But there is no empirical convergence analysis in the evaluation section e.g., training loss curves or number of FL rounds, etc. }

%\wajih{Add in the evaluation section how many clients we have in each dataset. "I can't find the number of FL clients in your text" }

Our evaluation experiments are conducted on a machine running Ubuntu 18.04.6 LTS, equipped with a 10-core Intel CPU, NVIDIA RTX 2080 GPU, and 120 GB of memory. In our experiments, we set the federated learning rounds and the number of categorized \gnnshort to 10 per host. Each model is trained for 20 epochs per round. We use regularization and dropout layers in our models to avoid overfitting. To evaluate \Sys, we address the following research questions: %\wajih{I have moved things around so align the RQs with that.}

%\wajih{RQ6 is not in this section. so please remove it. }

%\wajih{Add section references like I added for RQ1}

% \Fix{There are eight subsection in the evaluation while we have six research question. Make them consistent. This can confuse reader. Also, if you moved something to Appendix, refer them below. }

% \wajih{These are just too many research question. You need to combine similar RQs into one. If you think something is an ablation study then move it to the ablation study section. }
\begin{itemize}[leftmargin=*,itemsep=0.1em, parsep=0em, topsep=0em]
  \item \textbf{RQ1.} How does \Sys compare to vanilla privacy-preserving \pids in terms of detection performance? (Section~\ref{sub:detect:perf:vanilla})
  \item \textbf{RQ2.} How does \Sys compare to existing systems in terms of detection performance? (Section~\ref{sub:detect:perf})
  \item \textbf{RQ3.} How scalable is \Sys in an enterprise-level setting with multiple host machines? (Section~\ref{cost_metric})
  \item \textbf{RQ4.} How does \Sys compare to existing FL solutions in addressing data heterogeneity and non-IID challenges? (Section~\ref{sec:fedalternatives})
  \item \textbf{RQ5.} How robust is \Sys against various adversarial attacks?
  %\item \textbf{RQ5.} How effective is the Word2vec harmonization scheme for handling feature heterogeneity? (Appendix~\ref{sub:word2vec:harmonization:efficacy})
  %\item \textbf{RQ6.} How effective is the categorization-based \gnnshort ensemble for handling data heterogeneity? (Appendix~\ref{sub:categorized:learning:efficacy})
  
  % \item \textbf{RQ6.} How robust is \Sys against adversarial mimicry attacks?
  \item \textbf{RQ6.} What is the resource consumption of various components of \Sys and its end-to-end processing time on a client machine? (Appendix~\ref{sec:resource_consumption})
  \item \textbf{RQ7.} What does the ablation study reveal about \Sys's effectiveness across key factors like \gnnshort submodels, averaging rounds, federated averaging rounds, \wordvec harmonization and categorization-based \gnnshort ensemble? (Appendix~\ref{app:ablation})
  
  \end{itemize}  


\PP{Implementation} \Sys is developed in Python with around 5500 lines of code. It leverages the PyTorch and Torch Geometric libraries to implement the federated provenance graph learning framework. The graph learning framework uses the GraphSAGE~\cite{hamilton2017inductive} family of \gnnshort. Our architecture consists of two graph convolution layers with a Tanh activation function in between. The last layer uses a softmax function to output class probabilities for the nodes. For implementing the \wordvec model, we employ the Gensim library. Secure communication between clients and the utility server is ensured through Python's Cryptography module. The federated averaging, semantic vector harmonization, and entity categorization modules are implemented as individual Python functions on the central and utility servers.

% \PP{Datasets} We have utilized the \darpa E3~\cite{error3}, E5~\cite{bug5}, and \optc~\cite{darpaoptc} datasets for our evaluation. The E3 and E5 datasets consist of several adversarial engagements that simulate real-world APTs on enterprise networks. In these exercises, the red team aims to exploit vulnerabilities in the enterprise's services while hiding their attacks behind benign system activities. The logs captured from these exercises are documented under various scenario names, including Cadets, Trace, Theia and ClearScope. The \optc dataset, another open-source resource from \darpa, encompasses a comprehensive collection of audit logs from an enterprise environment with 1,000 hosts. This dataset includes six days of benign system logs, serving as training data for our system to learn normal behavior patterns. Subsequently, attack logs span three days of system activities, featuring red team tactics such as initial compromises, privilege escalations, malicious software installations, and data exfiltration. Each of these datasets is accompanied by ground truth documents that facilitate the distinction between benign and malicious events. For our evaluation, we employ attack labels from existing systems, such as \threatrace, \kairos and \flash for our evaluation. We evaluate our system using the same subset of datasets as existing open-source systems such as \kairos and \flash, and we utilize the attack node labels provided by them to ensure fairness.

\begin{table*}[!t]
  \centering
  \scriptsize
  \caption{Comparison of \Sys against vanilla privacy-preserving PIDS as baseline.}
  \setlength{\tabcolsep}{4pt} % slightly narrower columns
  \renewcommand{\arraystretch}{1} % optional: improves vertical spacing
  \begin{tabular*}{\textwidth}{@{\extracolsep{\fill}}lcccccccccccccc}
    \toprule
    \multirow{2}{*}{\textbf{Dataset}} &
    \multicolumn{7}{c}{\textbf{Baseline}} &
    \multicolumn{7}{c}{\textbf{\Sys}} \\
    \cmidrule(lr){2-8} \cmidrule(lr){9-15}
    & Precision & Recall & F1-score & TP & FP & FN & TN
    & Precision & Recall & F1-score & TP & FP & FN & TN \\
    \midrule
    E3-CADETS       & 0.64 & 0.99 & 0.78 & 12846 & 7243 & 4 & 699725
                    & \TCP & \TCR & \TCF & \TCTP & \TCFP & \TCFN & \TCTN \\
    E3-TRACE        & 0.85 & 0.99 & 0.92 & 67357 & 11390 & 20 & 2404623
                    & \TTP & \TTR & \TTF & \TTTP & \TTFP & \TTFN & \TTTN \\
    E3-THEIA        & 0.69 & 0.99 & 0.81 & 25314 & 11337 & 38 & 3493999
                    & \TTHP & \TTHR & \TTHF & \TTHTP & \TTHFP & \TTHFN & \TTHTN \\
    OpTC            & 0.36 & 0.99 & 0.53 & 649 & 1142 & 1 & 1286213
                    & \TOP & \TOR & \TOF & \TOTP & \TOFP & \TOFN & \TOTN \\
    E5-CADETS       & 0.76 & 0.99 & 0.86 & 40098 & 12356 & 10 & 1258638
                    & \ETCP & \ETCR & \ETCF & \ETCTP & \ETCFP & \ETCFN & \ETCTN \\
    E5-THEIA        & 0.73 & 0.99 & 0.84 & 54810 & 20767 & 270 & 1250910
                    & \ETTHP & \ETTHR & \ETTHF & \ETTHTP & \ETTHFP & \ETTHFN & \ETTHTN \\
    E5-ClearScope   & 0.79 & 0.97 & 0.88 & 13670 & 3491 & 397 & 79857
                    & \ETClP & \ETClR & \ETClF & \ETClTP & \ETClFP & \ETClFN & \ETClTN \\
    \bottomrule
  \end{tabular*}
  \label{summary:benchmarks:vanilla}
\end{table*}


\PP{Datasets} We have utilized the \darpa E3~\cite{error3}, E5~\cite{bug5}, and \optc~\cite{darpaoptc} datasets for our evaluation. These datasets contain real-world APT attacks observed in enterprise networks. The attack traces they include are stealthy, span several days, and mirror the characteristics of actual APTs. Consequently, achieving strong detection accuracy on these datasets indicates that our system can deliver comparable performance in real-world deployments. Furthermore, these datasets incorporate logs of various sizes from Linux, FreeBSD, Android, and Windows operating systems. Our system’s robust detection accuracy across all datasets demonstrates its effective generalization to heterogeneous platforms with differing log sizes, reaching performance levels on par with state-of-the-art centralized \pids. Notably, the datasets capture APT attacks of varying stealthiness, with the proportion of malicious nodes ranging from 0.05\% in \optc to 2\% in E3, and E5. The low infiltration rate in \optc underscores the system’s capacity for detecting highly stealthy adversaries, whereas the more widespread attacks in E3 and E5 highlight the resilience of our approach when confronted with a higher density of malicious nodes. Collectively, these datasets serve as a strong benchmark to evaluate the scalability and adaptability of our system. Each \darpa dataset is accompanied by ground truth documents that aid in distinguishing benign events from malicious ones. For this evaluation, we employ attack labels from existing systems such as \threatrace, \kairos, and \flash. Further details regarding the datasets appear in Appendix~\ref{sec:dataset:description}. ATLASv2~\cite{riddle2023atlasv2} is another recent dataset containing APT attack traces, but we did not evaluate it because it only includes data from two hosts, making it unrepresentative of a typical federated learning scenario.



\PP{Detectors for comparison} To benchmark our system, we compare against state-of-the-art \pids. \threatrace~\cite{wang2022threatrace} is a node-level system using graph representation learning to detect anomalous nodes in provenance graphs. MAGIC~\cite{jia2023magic} applies masked graph representation learning to identify threats. \flash~\cite{flash2024}, another node-level system, leverages semantic feature vectors and an embedding recycling database for enhanced detection and efficiency. As shown in Table~\ref{summary:benchmarks:large}, \flash surpasses  and thus serves as our primary baseline. \Fix{We also include \orthrus~\cite{jiang2025orthrus} and \kairos~\cite{cheng2023kairos}, which use temporal graph networks to capture system behavior over time.} We exclude Streamspot~\cite{streamspot}, Unicorn~\cite{han2020unicorn}, and \threatrace as they are outperformed by \flash and \kairos. While \disdet~\cite{dong2023distdet}, Prographer~\cite{yangprographer}, and Shadewatcher~\cite{shadewatcher} are notable, we exclude them as \disdet and Prographer are closed-source, and Shadewatcher relies on proprietary components, limiting reproducibility. Moreover, \flash and \orthrus have already demonstrated superior performance over Prographer~\cite{yangprographer} and Shadewatcher~\cite{shadewatcher}. We provide more details on why \disdet is unsuitable for comparison in Section~\ref{s:relwk}.

\Fix{It is important to note that, similar to existing works (e.g., \kairos, Shadewatcher, and Prographer), \Sys considers only three node types in provenance graphs: \emph{processes}, \emph{files}, and \emph{sockets}. However, in the E3 dataset, \flash has also been evaluated using additional node types. Therefore, we executed \flash using these three node types to report the results in Table~\ref{summary:benchmarks:large}.}

\subsection{RQ1: Detection Performance Against a Vanilla Privacy-Preserving PIDS}
\label{sub:detect:perf:vanilla}

%\wajih{Add a line about Vanilla Privacy-Preserving PIDS and how it was created by using basic FL with Flash. and say that Besides Flash we tried other SOTA PIDS and we got similar results for Vanilla Privacy-Preserving PIDS so we only include results for Flash.}

We conducted experiments to analyze the detection performance of \Sys in comparison to a vanilla privacy-preserving \pids, constructed by naively applying FL to an existing centralized \pids such as \flash. To simulate this setup, we operated \flash in a decentralized manner: each client locally trained \wordvec to encode semantic features and then trained their own \gnnshort models. These models were subsequently aggregated into a global model using the standard federated averaging algorithm. The results are shown in Table~\ref{summary:benchmarks:vanilla}. \Sys consistently outperforms the vanilla FL \flash across all detection metrics. This performance gain stems from \Sys's use of \wordvec harmonization and categorization-based ensemble learning, which effectively handle data heterogeneity. In contrast, the vanilla FL \flash lacks any mechanism to address this heterogeneity, resulting in degraded detection performance.



 \subsection{RQ2: Detection Performance Against SOTA PIDS}
 \label{sub:detect:perf}


\documentclass[conference]{./sty/IEEEtran-3}
\include{macros}
\include{table-macros/exp}

\begin{document}

\title{Private Yet Accurate: A Decentralized Approach to System Intrusion Detection}

\maketitle

\input{abstract}
% \input{concepts}
% \input{keyword}
\input{intro}
\input{relwk}
% \input{background}
\input{threat}
\input{design}
% \input{impl}
\input{eval}

\input{privacy}
% \input{adversarial}
\input{discussion}
\input{conclusion}
% \input{ethics}
% \input{ack}

% \balance
% \bibliographystyle{plain}
% {
% \footnotesize
% \setlength{\bibsep}{3pt}
% \bibliography{bib-files/hassan,bib-files/hassan-misc}
% }
% \input{appendix}

\balance
\bibliographystyle{IEEEtranS}
{
\footnotesize
\setlength{\bibsep}{3pt}
\bibliography{bib-files/hassan,bib-files/hassan-misc}}
\input{appendix}
\end{document}
% \wajih{do not discuss FP/FN in this subsection as I have removed them from the table.}
We conducted experiments to assess how \Sys compares with other systems in terms of detection performance. Initially, we outline our methodology for deploying \Sys on the \darpa E3, E5, and \optc datasets. The E3 dataset comprises various scenarios, including Cadets, Theia, and Trace, each representing logs generated by a single host machine. To evaluate \Sys on E3, we treat each scenario as an individual host. Consequently, in our federated learning approach, we trained local \gnnshort models on each scenario individually. These local models then participated in federated averaging, a process repeated across 10 rounds. Upon completing the training, we evaluated the global \gnnshort model against the attack logs from these E3 scenarios. Similar to E3, the E5 dataset is also divided into different scenarios; however, each scenario comprises three different hosts. For the \optc dataset, we used 10 hosts selected randomly for inclusion in our federated provenance graph learning experiment. Additionally, we conducted an enterprise-level analysis using all 1000 hosts from the \optc dataset, the results of which are detailed in Section~\ref{cost_metric}. For each dataset, we use logs from the benign period to train our system and then evaluate it on the attack logs. The attack logs follow the benign logs in the timeline. For example, the \optc dataset has six days of benign system logs and three days of attack logs. We run \Sys on all days of attack logs to analyze detection performance. For these evaluations, we use the same detection metrics as defined by existing node-level detectors such as \threatrace and \flash.

Table~\ref{summary:benchmarks:large} reveals that \Sys's performance on these datasets is comparable to that of \flash, despite the data heterogeneity, diverse log patterns, and data imbalance contained within each E3, E5, and \optc datasets. This underscores \Sys's capability to maintain robust detection performance amidst such heterogeneity. \kairos's evaluation, based on a coarser time-window granularity compared to the node-level granularity of \flash and \Sys, poses a challenge for direct comparison. Nevertheless, our results remain competitive with \kairos. Beyond detection performance, we also highlight \Sys's qualitative advantages, including its privacy-preserving features and decentralized, scalable operation. These aspects underscore the value of \Sys in contrast to centralized systems, emphasizing its high practicality in real-world deployments. Since \kairos was not originally evaluated on the DARPA E3 Trace, we did not compare \kairos on this dataset for fairness because unlike \flash extensive hyperparameter tuning might be needed for \kairos to produce the best results.

%  \subsection{Resource consumption}

%  We conducted experiments to analyze the resource consumption of the central, utility server, and client-side modules of \Sys. We modeled the resource utilization on a client machine using different batches of audit events of varying sizes. For the central and utility servers, we studied resource consumption by varying the number of clients to understand the demands of federated averaging and semantic vector harmonization. The results, depicted in Figure~\ref{fig:resource}, indicate that \Sys's resource consumption is moderate. Specifically, \Sys can process up to 100,000 audit events simultaneously while consuming less than 900 MB of memory and utilizing less than 20\% of CPU resources. This performance suggests that \Sys does not significantly burden the client machine, especially considering the typically low event throughput on such machines. Additionally, our analysis of the host data in the \optc dataset shows that, on average, each client generates approximately 100,000 audit log events within a three-hour period. For the central and utility servers, the resource usage is minimal, demonstrating that our architecture is scalable and suitable for large organizations with many clients.

%  \begin{figure}[!t]
%   \centering
%   \subfloat[CPU utilization client side.]{\includegraphics[width=0.20\textwidth]{fig/cpu.pdf}\label{cpu_client}}
%   \hfill
%   \subfloat[RAM utilization client side]{\includegraphics[width=0.20\textwidth]{fig/ram.pdf}\label{ram_client}}
%   \hfill
%   \subfloat[CPU utilization central server.]{\includegraphics[width=0.20\textwidth]{fig/cpu_central.pdf}\label{cpu_central}}
%   \hfill
%   \subfloat[RAM utilization central server]{\includegraphics[width=0.20\textwidth]{fig/ram_central.pdf}\label{ram_central}}
%   \hfill
%   \subfloat[CPU utilization utility server.]{\includegraphics[width=0.20\textwidth]{fig/cpu_utility.pdf}\label{cpu_utility}}
%   \hfill
%   \subfloat[RAM utilization utility server]{\includegraphics[width=0.20\textwidth]{fig/ram_utility.pdf}\label{ram_utility}}
%   \caption{Resource consumption of various components of \Sys.}
%   \label{fig:resource}
%   \vspace{-2ex}
% \end{figure}



% \subsection{Enterprise settings evaluation}
% \label{cost_metric}
% We examined the operational costs of deploying \Sys in comparison to centralized solutions like \flash and \kairos for organizational use. Our evaluation concentrates on two primary cost components: network expenses (\(C_{N}\)), and processing costs (\(C_{P}\)).

% \PP{Network Overhead:} We estimate this cost for the \flash and \kairos system using the \optc dataset. Each host within \optc produces approximately 1GB of audit logs daily, equating to nearly one million audit events. For an organization with 1,000 hosts, the total daily log volume would be 1,000 GB. This data volume necessitates transmission over the network to a central server operating the \pids. In contrast, \Sys achieves a significant reduction in these overheads. The only network expenses for \Sys arise from the transmission of the \gnnshort and \wordvec models; the \gnnshort model is roughly 13kb, while the average \wordvec model is 6 MB. Thus, the communication cost with the central server would be 12.70 MB, and for the utility server, it would be 5.86 GB. Ultimately, \Sys results in a 170-fold decrease in communication costs compared to \flash and \kairos.

% \PP{Processing Overhead:} Additionally, \flash processes one million events in about 100 seconds, implying that processing events from 1,000 clients would necessitate approximately 27.7 hours. In contrast, \kairos processes 57,000 events in 11.6 seconds, leading to a processing time of 204 seconds for one million events. Consequently, processing data from 1,000 clients with \kairos would require around 56.6 hours. Compared to existing systems, \Sys processes client logs in a decentralized manner and the total processing time is bounded by the client with the most log data; it will only take approximately 3 minutes to run inference on the complete \optc dataset. The central and utility servers conduct a simple mean operation on the models, taking only a few seconds.

% \PP{Overall Operating costs:} To calculate the daily operational costs of running these systems for a specific organization, we utilized the Google Cloud Platform's (GCP) pricing calculator~\cite{gcp}. The cost to operate \flash under the previously mentioned workload is estimated at \$135, with computing expenses accounting for \$100 and network costs comprising \$35. Given its longer processing duration, the operational cost for \kairos would be twice that of \flash. Compared to these systems, the operational cost of \Sys for 1,000 client machines would be approximately \$10 per day, marking a 13-fold reduction compared to \flash and a 26-fold reduction compared to \kairos.

% Scalability concerns are addressed in this section

\subsection{RQ3: Scalability in Enterprise settings}
\label{cost_metric}
We analyzed the performance of our system in an enterprise setting using the \optc dataset, which includes a large number of host machines. We compared our system to centralized systems such as \flash and \kairos, focusing on important metrics including network and processing overhead. \Fix{We also provide further discussion on scaling FL to massive enterprise networks in Appendix~\ref{state:explosion}.}

\PP{Network Overhead:} We estimate this cost for the \flash and \kairos system using the \optc dataset. Each host within \optc produces approximately 1GB of audit logs daily, equating to nearly one million audit events. For an organization with 1,000 hosts, the total daily log volume would be 1,000 GB. This data volume necessitates transmission over the network to a central server operating the \pids. In contrast, \Sys achieves a significant reduction in these overheads. The only network expenses for \Sys arise from the transmission of the \gnnshort and \wordvec models; the \gnnshort model is roughly \modelsize, while the average \wordvec model is 6 MB. Thus, the communication cost with the central server would be 12.70 MB, and for the utility server, it would be 5.86 GB. Hence, the network latency of model communications in \Sys is minimal compared to centralized systems where the raw system logs need to be sent over the network. Ultimately, \Sys results in a 170-fold decrease in communication costs compared to centralized \pids.  

\PP{Processing Overhead:} Additionally, \flash processes one million events in about 100 seconds, implying that processing events from 1,000 clients would necessitate approximately 27.7 hours. In contrast, \kairos processes 57,000 events in 11.6 seconds, leading to a processing time of 204 seconds for one million events. Consequently, processing data from 1,000 clients with \kairos would require around 56.6 hours. Compared to existing systems, \Sys processes client logs in a decentralized manner and the total processing time is bounded by the client with the most log data; it will only take approximately 3 minutes to run inference on the complete \optc dataset. The central and utility servers conduct a simple mean operation on the models, taking only a few seconds.

\subsection{RQ4: Comparison with existing FL solutions for heterogeneity.}
\label{sec:fedalternatives}

We conducted experiments to evaluate the efficacy of existing federated learning solutions that address data heterogeneity and the non-IID problem. Specifically, we examined FedProx~\cite{li2020federated} and FedOpt~\cite{asad2020fedopt}. FedProx mitigates client heterogeneity by adding a proximal term to the local loss function, penalizing deviations from the global model and thereby reducing the impact of statistical heterogeneity. In contrast, FedOpt uses a server-side optimizer to aggregate updates from distributed clients, enhancing convergence. We compared these techniques with standard FedAvg and the \Sys variant of FedAvg, which incorporates semantic harmonization and categorization-based GNN ensemble learning. We used the \optc, E3, and E5 datasets to conduct this experiment. Table~\ref{fedoptprox} presents the evaluation results. We averaged the results for E3 and E5 across individual scenarios (Cadets, Theia, ClearScope) due to space constraints. Both FedProx and FedOpt outperform FedAvg; however, neither matches the performance of \Sys.

FedProx's penalty on global and local deviations focuses the model on patterns common across all clients, preventing it from learning client-specific patterns. Consequently, using this global model for anomaly detection on individual clients often yields numerous false alarms and poor detection performance. Moreover, while FedOpt can effectively aggregate model updates using an optimizer rather than simple averaging, updates that are highly disparate, a common scenario in system logs due to diverse client activities, cause FedOpt to fail in harmonizing the updates. As a result, it loses information on important benign patterns, leading to low precision and recall.

In contrast, our system employs an ensemble learning framework. First, it categorizes system patterns into standardized bins via process entity categorization. Next, \gnnshort submodels learn the data distribution within each bin. Submodels with similar learned distributions are then averaged, preventing the conflation of unique client patterns and improving precision. Moreover, our semantic vector harmonization reduces heterogeneity in the \gnnshort feature vectors, making the updates less disparate and improving model convergence in comparison to other techniques.


\begin{table}[!t]
  \centering
  \scriptsize
  \caption{Federated averaging algorithms comparison.}
  \setlength{\tabcolsep}{1.6pt}
  \begin{tabular}{cccccccccc}
    \toprule
    \multirow{2}{*}{\textbf{Methods}} & \multicolumn{3}{c }{\textbf{\optc}} & \multicolumn{3}{c }{\textbf{E3}} & \multicolumn{3}{c }{\textbf{E5}} \\
    \cmidrule(r){2-4} \cmidrule(r){5-7} \cmidrule(r){8-10}
    & {\bf Prec.} &  {\bf Rec.} & {\bf F-score} & {\bf Prec.}  & {\bf Rec.} & {\bf F-score} & {\bf Prec.}  & {\bf Rec.} & {\bf F-score} \\
    \midrule

    FedAvg & 0.36 & 0.99 & 0.53  & 0.73 & 0.99 & 0.84 & 0.76 & 0.98 & 0.86 \\
    FedProx & 0.48 & 0.95 & 0.64 & 0.79 & 0.98 & 0.87 & 0.82 & 0.95 & 0.88 \\
    FedOpt & 0.52 & 0.90  & 0.74 & 0.81 & 0.97 & 0.88 & 0.84 & 0.92 & 0.87 \\
    {\bf \Sys} & \TOP & \TOR & \TOF & 0.96 & 0.99 & 0.97 & 0.99 & 0.96 & 0.97 \\
    \bottomrule
  \end{tabular}
\label{fedoptprox}
\end{table}


% \subsection{Robustness against mimicry attacks}
% \label{sec:mimicry}

% We assessed the robustness of our system against the adversarial mimicry attack proposed by Goyal et al.~\cite{goyal2023sometimes} using the E3 dataset. The attack's objective is to generate embeddings for nodes in the attack graphs that are similar to those in the benign graph, thus evading detection. To achieve this, structures from benign nodes are integrated into the attack graph. Figure~\ref{mimicryattack} illustrates the experimental results, with the x-axis representing the number of benign structures added and the y-axis corresponding to anomaly scores for all attack nodes. The findings indicate that our detector maintains robustness against this attack; adding benign structures has minimal effect on the anomaly scores of the attack nodes. The superior robustness of \Sys can be attributed to our process-categorization-based ensemble \gnnshort architecture, wherein each model specializes in understanding the behavior of specific system entities. Thus, when the graph structure surrounding these entities changes, the models can easily detect these alterations.\footnote{In contrast, the \flash system relies on a downstream model using \gnnshort embeddings to detect malicious entities. Generalizing a single model across system entities can cause it to overlook critical details, making it vulnerable to mimicry attacks. As demonstrated by the authors, \flash initially experiences a drop in anomaly scores, increasing the likelihood of attack nodes evading detection, a vulnerability not present in our system.}

% \begin{figure}[!t]
%   \centering
%   \includegraphics[width=0.4\textwidth]{fig/adversarial.pdf}
%   \caption{Adversarial mimicry attack analysis.}
%   \label{mimicryattack}
%   \vspace{-2ex}
% \end{figure}\


% \subsection{Effect of differential privacy on accuracy}

% Differential privacy is a method that can be combined with federated learning to offer protection against inference attacks~\cite{lyu2020threats,nasr2019comprehensive,zari2021efficient} at the cost of detection accuracy. We have analyzed the robustness of our system against these attacks in Section~\ref{privacy}. Here we will examine the impact of differential privacy noise levels on the detection performance of \Sys. Differential privacy achieves model privacy by adding a controlled amount of random noise to the model parameters, which helps in concealing the influence of any individual data point. In our context, differential privacy introduces a parameter called epsilon $\epsilon$ which dictates the intensity of noise added to the local \gnnshort model updates before they are aggregated at the central server for federated averaging, as detailed in Section~\ref{sec:methodology}.

% The parameter $\epsilon$ is crucial; it is inversely related to the amount of noise added -- lower values of $\epsilon$ result in higher noise levels, thereby increasing privacy but potentially degrading the utility of the model. Conversely, a higher $\epsilon$ indicates less noise, which may improve the model's detection capabilities but reduce privacy protection. By adjusting $\epsilon$ during the training phase, we assess the trade-off between privacy and detection performance in the globally trained models across various $\epsilon$ settings. Figure~\ref{epsvsscore} shows that increasing noise strength degrades model utility offering more privacy at the expense of reduced accuracy.

% \begin{figure}[!t]
%   \centering
%   \includegraphics[width=0.25\textwidth]{fig/epsvsscore.pdf}
%   \caption{Effect of differential privacy noise on detection using E3 dataset. Note that we observed similar results on the other datasets.}
%   \label{epsvsscore}
%   \vspace{-2ex}
% \end{figure}


% \begin{figure*}[!t]
%   \centering
%   \subfloat[Anomaly threshold effect.]{\includegraphics[width=0.24\textwidth]{fig/thresh.pdf}\label{thresh}}
%   \hfill
%   \subfloat[Effect of number of categories vs detection performance using E3 dataset.]{\includegraphics[width=0.24\textwidth]{fig/kvsscore.pdf}\label{catgvsscore}}
%   \hfill
%   \subfloat[Federated averaging rounds vs detection performance using E3 dataset.]{\includegraphics[width=0.24\textwidth]{fig/roundsvsscore.pdf}\label{roundsvsscore}}
%   \hfill
%   \subfloat[Effect of number of hosts vs detection metrics using \optc dataset]{\includegraphics[width=0.24\textwidth]{fig/scoresvshosts.pdf}\label{rscoresvshosts}}
%   \caption{Ablation Study of various \Sys components.}
%   \label{ablation}
%   \vspace{-2ex}
% \end{figure*}


\section{Privacy Preservation Analysis}
\label{sec:privacy}

%\wajih{Please check existing papers to see if we can add formalism to this subsection. This concern has been raised by several reviewers that our privacy analysis is not robust.}

%\wajih{ There was a comment/review about membership inference for unseen words. Is that addressed below?} Yes

% \wajih{We need to make our privacy analysis more formal. Look into  a privacy game where the adversary attempts to distinguish between two datasets. }

% \wajih{The private FL papers have a lot theorems, we just need to tailor them for our.}

% \wajih{Also we need a theorem for your Privacy Against Model Update Inference. There are two parts here. Define Theorem and then provide proof for that theorem.}

% \wajih{The privacy analysis seemed too informal. Simply stating that the cardinality of the token set is high to prove that the system is private is not very convincing. There were no empirical experiments as well.}

% \wajih{Is there a risk of bin-level inference. I think one reviewer said something about that. For example, if an attacker observes which bin a process was mapped to (e.g., “bin 3 → sshd-like processes”), this could still leak meta-information even without raw data. Can we discuss this attack in the privacy analysis section as well.}

In this section, we analyze the preservation of user privacy within \Sys, which is structured around three primary components: the central server, the utility server, and clients. The risk to \Sys's privacy arises from the possibility of inference attacks using model weights at the central server and \wordvec tokens at the utility server. These attacks aim to infer whether some system entity or attributes were used in the training data or not. We will discuss the scope and limitations of these attacks on \Sys below:

\PP{Component Roles} The central server's role involves the application of federated averaging to the \gnnshort models received from clients. The utility server performs contextual aggregation of semantic attribute vectors derived from clients' \(\wordvec\) models.  The mathematical representation for averaging vectors of a token \(k\) across \(N\) clients is given by: \( \bar{v}_k = \frac{1}{N} \sum_{i=1}^{N} v_{k,i} \) where \(\bar{v}_k\) is the averaged vector for token \(k\), and \(v_{k,i}\) is the vector representation of token \(k\) from the \(i\)-th client model. Clients are tasked with training the \wordvec and \gnnshort models on provenance graphs, these graphs are constructed from their local system audit logs.

\PP{Privacy Risk Analysis} Within \Sys, potential privacy compromises arise if either the central or utility server can infer specific details about individual clients' logs, such as the applications in use or particular attributes like filenames and IP addresses. Despite the central server's inability to access raw client data directly, it receives model updates from clients, thereby introducing a vulnerability to model inference attacks through analysis of these updates.

% Consider an attacker's objective to ascertain whether a system entity \(x\) with attributes \(y\) was utilized in training a client model \(m\). This necessitates the generation of multiple candidate node features for \(x\), taking into account various graph structures and interactions with other entities while considering diverse attributes. The search space for this task, \(S\), is extensive, spanning all conceivable processes, files, and network IPs.

% Assuming the server generates multiple candidate structures, it then requires access to the specific client's \(\texttt{\wordvec}\) model to generate feature vectors for these structures---a step prevented by the model's unavailability and inherent algorithmic randomness, rendering each training iteration of the \(\texttt{\wordvec}\) model distinct:
% \( F(x) = \texttt{\wordvec}(s_x) \)
% where \(F(x)\) is the feature vector of structure \(x\) and \(s_x\) is the candidate sequence.

\textbf{Central Server Privacy Theorem.}  
For any two datasets \(D_1\), \(D_2\) differing in the presence of a single structured sequence involving an entity \(x\), the probability that an adversary \(\mathcal{A}\), observing only the model updates \(\text{Enc}(D)\) received by the central server, can determine which dataset was used deviates from random guessing by at most a negligible function \(\epsilon(n)\), where \(n\) is the security parameter representing the randomness complexity of the client-side training process.

\textbf{Central Server Privacy Proof.}  
Let \(D_1\) and \(D_2\) be two datasets differing only in a structured sequence \(s_x\) involving entity \(x\). Each client encodes sequences using a randomized local embedding model \(\texttt{\wordvec}\), which maps \(s_x\) to a feature vector \(F(x) = \texttt{\wordvec}(s_x)\). These embeddings are then used to train the local model, whose updates are sent to the central server.

We formally model the encoding process \(\text{Enc}(D; r)\) as a randomized algorithm that takes as input a dataset \(D\) and a random seed \(r \in \{0,1\}^n\), where \(n\) is the security parameter. This seed governs all sources of local randomness, including embedding initialization, mini-batch sampling, and optimizer dynamics. 

Since training is stochastic and local models are never shared, two executions on the same dataset may produce statistically different updates. Let \(P_1\) and \(P_2\) denote the distributions over model updates \(\text{Enc}(D_1)\) and \(\text{Enc}(D_2)\), respectively. Their statistical distance is defined as:
\[
\Delta(P_1, P_2) = \sup_{S} \left| \Pr[\text{Enc}(D_1) \in S] - \Pr[\text{Enc}(D_2) \in S] \right|,
\]
where the supremum is taken over all measurable subsets \(S\) of the output space. Because \(\text{Enc}(D; r)\) depends on \(n\) bits of internal randomness—and because \(\mathcal{A}\) has no access to this randomness or to the client’s embedding model—it cannot reliably infer the presence of \(s_x\) in the training data.

Furthermore, the set of plausible structured sequences involving node \(x\), denoted \(\mathcal{S}(x)\), grows combinatorially with the graph topology and attribute space. Specifically, for a graph \(G = (V, E)\), a node attribute vocabulary \(\mathcal{A}_v\), and edge attribute vocabulary \(\mathcal{A}_e\), the number of candidate sequences in the \(\ell\)-hop neighborhood \(\mathcal{N}_\ell(x)\) satisfies:
\[
|\mathcal{S}(x)| \in \mathcal{O}\left(|\mathcal{A}_v|^{|\mathcal{N}_\ell(x)|} \cdot |\mathcal{A}_e|^{|\mathcal{N}_\ell(x)|}\right).
\]
Thus, even if the adversary attempts to enumerate candidates, it must simulate an exponential number of possibilities without access to the true embedding function \(\texttt{\wordvec}\).

As a result, the adversary’s distinguishing advantage—defined as the deviation from random guessing—is bounded by:
\[
\left| \Pr[\mathcal{A}(\text{Enc}(D_1)) = 1] - \tfrac{1}{2} \right| \leq \epsilon(n),
\]
where \(\epsilon(n)\) is negligible in the security parameter \(n\); that is, for every polynomial \(p(n)\), there exists an \(n_0\) such that for all \(n > n_0\), \(\epsilon(n) < \frac{1}{p(n)}\). Therefore, \Sys guarantees indistinguishability-based privacy against chosen-dataset inference attacks under server-side observation.

% Formally, the system \Sys is designed to ensure that the probability that an adversary can distinguish between two arbitrary system log datasets based on the observed updates is negligible:
% \begin{align*}
% \text{Attack}(D_1, D_2) &= \left| \Pr[\mathcal{A}(\text{Enc}(D_1)) = 1] \right. \nonumber \\
% &\quad - \left. \Pr[\mathcal{A}(\text{Enc}(D_2)) = 1] \right| \\
% &\leq \epsilon \nonumber
% \end{align*}
% where \(\epsilon\) is a negligible value, \(\text{Enc}(D)\) denotes the encoding of the dataset into model updates, and \(\mathcal{A}\) represents the adversary.

% To further quantify the security of \Sys in the context of semantic embeddings at the utility server, consider the scenario where an attacker has access to the \wordvec embeddings but not the original attributes used for generating them. Suppose the attacker attempts to reverse engineer the embedding vectors to recover the original system log data. The complexity, randomness and high dimensionality of the embedding space, combined with the non-linear transformations typically applied during \wordvec embedding process, ensure that the probability of successfully identifying the original tokens from the embeddings is negligibly small:
% \[
% \Pr[\text{Rev}(\mathcal{E}, e) = t] \leq \frac{1}{|T|}
% \]
% where \(\text{Rev}\) denotes the hypothetical reverse mapping function from embeddings to tokens, \(\mathcal{E}\) represents the embedding process, \(e\) is the observed embedding, \(t\) is the original token, and \(|T|\) is the cardinality of the token set. This probability indicates that correctly guessing the original token from its embedding is as likely as randomly selecting one token out of the entire set of possible tokens, which is practically infeasible given a sufficiently large and diverse token set.

% Consequently, the dual-server architecture and semantic featurization significantly limit the central server's capacity for inference attacks. The utility server's function of contextually aggregating semantic attribute vectors introduces another potential privacy concern if the server could deduce the entities these vectors represent. This risk is mitigated by the secure encryption of tokens using keys generated by the central server, which the utility server cannot access, ensuring privacy protection.

\textbf{Utility Server Privacy Theorem.}  
Let \(\mathcal{E}: T \to \mathbb{R}^d\) denote the client-side embedding function that maps input tokens \(t \in T\) to high-dimensional vectors \(e \in \mathbb{R}^d\). In \Sys, each client independently applies \(\mathcal{E}\) to locally generate semantic embeddings and encrypts all tokens before transmission. The utility server receives only encrypted tokens and their corresponding embeddings. Then, for any embedding \(e\) observed by the utility server, the probability that an adversary \(\mathcal{A}\) can correctly identify the original token \(t\) such that \(e = \mathcal{E}(t)\) is at most \(\frac{1}{|T|} + \epsilon(d)\), where \(\epsilon(d)\) is negligible in the embedding dimension \(d\).

\textbf{Utility Server Privacy Proof.}  
In \Sys, each client applies a local embedding function \(\mathcal{E}\) (e.g., word2vec) to convert input tokens \(t \in T\) into semantic vectors \(e = \mathcal{E}(t) \in \mathbb{R}^d\), and encrypts the original tokens before sending them to the utility server. Thus, the utility server receives only encrypted identifiers and their associated semantic embeddings, with no access to the plaintext tokens or the internals of \(\mathcal{E}\).

The embedding function \(\mathcal{E}: T \to \mathbb{R}^d\), trained using a distributional objective (skip-gram with negative sampling), is inherently non-injective. Its goal is to preserve contextual similarity, not one-to-one invertibility. Formally, there exist distinct tokens \(t_1, t_2 \in T\), with \(t_1 \neq t_2\), such that \(\mathcal{E}(t_1) \approx \mathcal{E}(t_2)\). Consequently, the preimage set of a given embedding \(e\) is defined as:
\[
\mathcal{E}^{-1}(e) = \{t \in T \mid \mathcal{E}(t) \approx e\},
\]
which may contain multiple semantically related tokens. This ambiguity arises because \(\mathcal{E}\) typically involves non-linear transformations and dimensionality reduction, meaning the mapping from \(T\) to \(\mathbb{R}^d\) is many-to-one and optimized for semantic coherence rather than reversibility. Hence, attempting to recover a unique \(t\) from \(e\) constitutes solving an ill-posed inverse problem over an equivalence class of plausible tokens.

Let \(\text{Rev}(\mathcal{E}, e)\) denote a hypothetical reverse function attempting to infer \(t\) from an embedding \(e\). Since the utility server has no access to the original training data or the encryption keys protecting the token identities, its best strategy is effectively bounded by random guessing over the token vocabulary:
\[
\Pr[\text{Rev}(\mathcal{E}, e) = t] \leq \frac{1}{|T|} + \epsilon(d),
\]
where \(\epsilon(d)\) represents any residual advantage due to structure in the embedding space. As the embedding dimension \(d\) increases, the space becomes more entangled and the function \(\epsilon(d)\) becomes negligible. That is, for every polynomial \(p(d)\), there exists \(d_0\) such that for all \(d > d_0\), \(\epsilon(d) < \frac{1}{p(d)}\).

Therefore, \Sys guarantees that the probability of successfully reverse-engineering semantic embeddings at the utility server is negligible, ensuring strong semantic privacy under adversarial observation.

\textbf{Client Level Privacy Theorem.}  
Let \(\theta_g\) be the global model obtained by aggregating updates from a set of clients \(\{C_1, C_2, \ldots, C_N\}\), where each client observes only a disjoint subset of application-specific semantic attribute vectors. Suppose a malicious client \(C_{\text{adv}}\) attempts to infer whether a particular application token \(t \in T\) used by another client is present in the global model. Then, under standard aggregation (e.g., FedAvg) and semantic vector disjointness across clients, the probability that \(C_{\text{adv}}\) can infer the presence of \(t\) is bounded by:
\[
\Pr[\text{Infer}(\theta_g, t) = 1] \leq \frac{|T_{\text{adv}}|}{|T|} + \epsilon(m),
\]
where \(T_{\text{adv}}\) is the set of semantic tokens accessible to \(C_{\text{adv}}\), \(|T|\) is the total token vocabulary, and \(\epsilon(m)\) is negligible in the number of clients \(m = |T \setminus T_{\text{adv}}|\) contributing disjoint attributes to the global model.

\textbf{Client Level Privacy Proof.}  
The global model \(\theta_g\) is constructed via aggregation of local updates from multiple clients, each of which computes gradients or embeddings based on their local token set. Let \(T\) denote the full organizational token vocabulary, and let \(T_{\text{adv}} \subset T\) denote the subset observed by the malicious client \(C_{\text{adv}}\).

Client \(C_{\text{adv}}\) observes \(\theta_g\), which encodes statistical signals from all \(T\), but without visibility into which updates correspond to which tokens. The adversary’s goal is to determine whether a specific target token \(t \in T \setminus T_{\text{adv}}\) was present in training. Since \(\theta_g\) is the result of averaging over model updates and \(\mathcal{E}(t)\) (the token's embedding) is not known to \(C_{\text{adv}}\), the adversary cannot attribute global weight shifts to tokens it cannot represent.

We model the attacker’s inference attempt as a binary decision: \(\text{Infer}(\theta_g, t) = 1\) if it believes \(t\) is present. Without access to semantic vectors for tokens outside \(T_{\text{adv}}\), the best strategy for \(C_{\text{adv}}\) is random guessing over the unknown token space \(T \setminus T_{\text{adv}}\). Hence:
\[
\Pr[\text{Infer}(\theta_g, t) = 1] \leq \frac{|T_{\text{adv}}|}{|T|} + \epsilon(m),
\]
where \(\epsilon(m)\) accounts for any statistical leakage from partial overlaps in tokens across clients. Since the number of clients \(m\) contributing disjoint tokens increases, \(\epsilon(m)\) becomes negligible — i.e., for every polynomial \(p(m)\), there exists \(m_0\) such that for all \(m > m_0\), \(\epsilon(m) < \frac{1}{p(m)}\).

Therefore, in \Sys, the absence of full semantic visibility ensures that malicious clients cannot meaningfully attribute specific applications or tokens to others via global aggregation, preserving privacy under attribution-based inference attacks.

Therefore, the architecture of \Sys robustly defends against model inference attacks, affirming the system's capacity to preserve privacy. This resilience is predicated on the non-collusion assumption between the central and utility servers—a standard premise upheld by related works~\cite{roy2020crypte,wu2022federated}, ensuring the system's high degree of privacy preservation.


% % \PP{Adversarial Attacks Analysis} We discussed robustness to mimicry attacks in Section~\ref{sec:mimicry}. Here, we address vulnerabilities to other types of adversarial attacks, including gradient-based attacks, model poisoning, and inference attacks (discussed in Section~\ref{sec:privacy}). Addressing these attacks is beyond the scope of this paper, as we focus on establishing an end-to-end framework for privacy-aware PIDS. We plan to integrate advancements in adversarial defense mechanisms against such attacks into \Sys to enhance its robustness in future work. These integrations are discussed below.

% Gradient-based adversarial attacks~\cite{chakraborty2021survey} typically require white-box access to the target machine learning model, including its parameters. This necessity often renders them impractical for real-world applications. In contrast, black-box attacks, which utilize iterative, query-based techniques, tend to be more detectable and complex to implement due to their conspicuous nature. Such attacks are feasible if an attacker manages to compromise a client machine. However, during the operational phase, a compromised client cannot affect other clients because they are working independently. Several existing defenses can be employed during model training to enhance the system's resilience against these attacks. Adversarial training~\cite{tramer2019adversarial} is one effective strategy, wherein the model is trained with perturbed input data to increase its robustness to such attacks.

% During the training phase, poisoning attacks executed by malicious actors may introduce corrupt weights to compromise the global model~\cite{jagielski2018manipulating}. To improve \Sys's resilience against such threats, several defensive methods can be used. Among these, advanced model aggregation methods, such as Multi-Krum~\cite{munoz2019byzantine}, can be particularly effective. This method employs clustering techniques on the central server to identify anomalous updates during model aggregation. Consequently, outlier updates are removed, enhancing the system's robustness against poisoning.

\section{Adversarial Attacks Analysis}
\label{sec:adversarial}

\begin{figure}[!t]
  \centering
  \includegraphics[width=0.34\textwidth]{fig/adversarial.pdf}
  \caption{Adversarial mimicry attack analysis.}
  \label{mimicryattack}
  \vspace{-2ex}
\end{figure}


In this section, we analyze the robustness of our system against mimicry, model poisoning, and gradient-based attacks. Membership inference attacks are discussed in Section~\ref{sec:privacy}.

\PP{Mimicry Attacks}  We evaluated our system's resilience against the adversarial mimicry attack detailed by Goyal et al.~\cite{goyal2023sometimes} and compared its robustness with \flash. The attack aims to mimic benign graph embeddings, integrating benign node structures to evade detection. Using the E3 dataset, similar to \flash for fair comparison, our findings (shown in Figure~\ref{mimicryattack}) reveal that our detector remains robust; the integration of benign structures has a minimal impact on anomaly scores. Our system's superior robustness is due to our process-categorization-based ensemble \gnnshort architecture, which allows model specialization for different system entities, enhancing detection of structural changes. Conversely, \flash, relying on a generalized model for anomaly detection, exhibits vulnerabilities to mimicry attacks, as it may miss critical details. As demonstrated by the authors, \flash initially experiences a drop in anomaly scores, increasing the likelihood of attack nodes evading detection—a vulnerability not present in our system. \footnote{PROVNINJA~\cite{mukherjee2023evading} is another mimicry attack that uses benign process profiles to find adversarial evasion strategies. It faces challenges against \pids like \flash and \Sys due to their multimodal architecture, which combines \wordvec for featurization and \gnnshort for anomaly detection. This complexity makes it difficult for attackers to create effective evasion strategies without accessing \wordvec directly or excessively querying the model, which conflicts with the black box nature of the assumed threat model of PROVNINJA.}



\PP{Model Poisoning Attacks} These occur when one or more malicious clients submit corrupted local model updates to the central server during training, thereby degrading the model’s performance. Existing methods typically assume an attack-free training phase, providing no defense against such attacks. We conducted experiments on the \optc dataset to assess the impact of poisoning attacks on our system, and we also evaluated how robust aggregation methods—such as Multi-Krum~\cite{munoz2019byzantine}—can enhance resilience.

Multi-Krum compares each client’s gradient with those of other clients, retaining only the most consistent updates. Since malicious updates must deviate significantly from benign ones to degrade performance, Multi-Krum effectively identifies and discards them as outliers. Figure~\ref{fig:poison} shows our experimental results using both FedAvg and Multi-Krum. With simple federated averaging, malicious noise critically affects the model by disrupting the benign distribution it learns. In contrast, Multi-Krum isolates and removes erroneous updates from malicious clients, preserving the global model. Notably, Multi-Krum assumes that fewer than one-third of all clients are malicious; therefore, in our experiments, we tested with a maximum of 30\% compromised clients.

\begin{figure}[!t]
    \centering
    \subfloat[FedAvg aggregation.]{\includegraphics[width=0.23\textwidth]{fig/FedAvg_poisoning.pdf}\label{fedavgpoison}}
    \hfill
    \subfloat[Multi-Krum aggregation ]{\includegraphics[width=0.23\textwidth]{fig/multi-krum.pdf}\label{multikrumpoison}}
    \hfill
    \caption{Model poisoning attack analysis.}
    \label{fig:poison}
    \vspace{-3ex}
  \end{figure}

\PP{Gradient-based Attacks} These attacks~\cite{chakraborty2021survey} exploit detailed knowledge of the target model, including its architecture and parameters, to calculate and apply malicious perturbations. Such white-box access allows an attacker to compute precise gradients that indicate how inputs should be modified to degrade the model’s performance. Other attacks tend to be black-box in nature, relying on iterative, query-based techniques to influence the model’s decisions. However, these repeated computations and queries run counter to the attacker’s aim of remaining inconspicuous, as they generate substantial activity and leave a significant footprint. Consequently, such attacks are often impractical in real-world scenarios. Several defenses can be employed during model training to bolster the system’s resilience against these threats. Adversarial training~\cite{tramer2019adversarial} is one effective strategy in which the model is trained on perturbed input data, thereby increasing its robustness against these attacks.

\subsection{Effect of differential privacy on accuracy}
\label{app:dp}

In Section~\ref{sec:privacy}, we analyze how our system design offers strong guarantees against model inference attacks. Here we discuss another alternative technique for preserving privacy through the use of Differential Privacy (DP). It can be integrated with federated learning to protect against inference attacks~\cite{lyu2020threats,nasr2019comprehensive,zari2021efficient}, though it comes at the cost of detection accuracy.

Differential privacy safeguards the model by injecting a controlled amount of Gaussian noise into its parameters, thereby concealing the influence of any individual data point. In our work, we adopt a node-level DP strategy, which ensures that each node’s features and labels are protected when noise is added to the gradient updates. As a result, individual node contributions remain obscured in the aggregated model parameters, reducing the likelihood of identifying specific nodes or their features. We conduct experiments to examine how varying levels of differential privacy noise affect the detection performance of \Sys. The noise is adjusted based on a privacy budget defined by~$\epsilon$. This noise is applied to local \gnnshort model updates before they are aggregated at the central server during federated averaging, as described in Section~\ref{sec:methodology}.

\begin{figure}[!t]
  \centering
  \includegraphics[width=0.3\textwidth]{fig/epsvsscore.pdf}
  \caption{Effect of differential privacy noise on detection using E3 dataset. Note that we observed similar results on the other datasets.}
  \label{epsvsscore}
  \vspace{-2ex}
\end{figure}


The privacy budget~$\epsilon$ plays a pivotal role. Lower values of $\epsilon$ increase the noise injected into the model, which enhances privacy at the expense of detection performance. Conversely, higher values of $\epsilon$ inject less noise, improving detection accuracy but offering weaker privacy guarantees. By tuning $\epsilon$ during training, we explore the balance between privacy and detection effectiveness across various settings. As shown in Figure~\ref{epsvsscore}, increasing the noise level (i.e., reducing $\epsilon$) degrades the model’s utility, thereby providing more privacy at the cost of lower accuracy.

Hence, although DP is a valid solution for protecting privacy, the accuracy deterioration that comes with it reduces its utility in the domain of intrusion detection, where high detection performance is needed. Due to these reasons, we do not use DP in \Sys and instead design a dual-server architecture to offer privacy guarantees without adding noise to the model updates.

\label{sec:discussion}

%\wajih{Use macros for common words in this whole section.}

%\wajih{Be consistent with the terminology that you used in the abstract/intro.}

%\wajih{Move adversarial attacks in a separate section.}

\PP{Concept drift} This is a problem where the data distribution of the underlying system evolves over time. For instance, with the emergence of new system activities, the patterns learned by \Sys during training might not remain valid. This drift could lead to misclassifications, as new benign behaviors might be mistakenly identified as anomalies. One mitigation strategy for this involves periodic retraining with more recent data to update the models. Techniques mentioned in recent works~\cite{lu2018learning, barbero2022transcending,jordaney2017transcend} can be leveraged to deal with the problem of concept drift. %\wajih{we need some citations here. Read Kairos and prographer paper. Maybe there have more information about this.}

\PP{Alert investigation} \Sys performs anomaly detection on the local provenance graph of each client. However, similar to existing IDSes~\cite{flash2024,cheng2023kairos,wang2022threatrace}, our system is also susceptible to generating false alarms. As a result, it becomes essential to analyze these false alarms. Currently, the task of analyzing these alarms falls upon individual users. \Sys notify users about detected alerts, enabling them to manually review the activities associated with these alert nodes to ascertain whether they are a real threat or a false positive. We identify privacy-preserving alert verification as a promising research direction. We leave it to future work to develop methods for privately sharing alert data with a central server, enabling security analysts to perform more in-depth attack analysis.

%\wajih{Investigation using \Sys. We need to add one paragraph saying something about how to do an investigation using \Sys after detection. It does not have to be detailed. We just need to show that we thought about this problem. At the end of the paragraph say that we leave this for future work.}

%\wajih{Talk about why we did not consider other privacy-preserving techniques beyond the federation, such as secure multi-party computation or homomorphic encryption.}
\section{Conclusion}
\label{sec:conclusion}

We introduced \Sys, a privacy-preserving intrusion detection system that combines federated provenance graph learning with secure \wordvec harmonization and process entity categorization-based \gnnshort ensemble training. \Sys addresses data heterogeneity, preserves client privacy, reduces network overhead, and scales efficiently. Evaluations on open-source datasets show that \Sys matches centralized \pids in accuracy, scales better, and remains robust against adversarial attacks. \wajih{Clean up references. Ensure that there are no duplicates. Also make sure that we don't have page numbers, doi, in the citations.}
% \section{Ethics Considerations and Compliance with the Open Science Policy}
\label{sec:ethics}

In conducting our research, we strictly adhere to all ethical guidelines and the USENIX open science policy. Below are the key ethical considerations:

\begin{itemize}[leftmargin=*,itemsep=0.1em, parsep=0em, topsep=0em]
\item Our research does not identify any new vulnerabilities.
\item We conduct all experiments in a controlled environment using open-source datasets, ensuring no live systems are tested without informed consent.
\item Our activities do not violate any terms of service and involve no human subjects. 
\item Our research has not negatively impacted any team members and complies with all relevant laws. \item We are committed to transparency, making our research, systems, scripts, and datasets available to the public to enable reproduction and extension of our work. 
\end{itemize}

% \input{ack}

% \balance
% \bibliographystyle{plain}
% {
% \footnotesize
% \setlength{\bibsep}{3pt}
% \bibliography{bib-files/hassan,bib-files/hassan-misc}
% }
% \section{Appendix}
\label{sub:hyper}

  



\balance
\bibliographystyle{IEEEtranS}
{
\footnotesize
\setlength{\bibsep}{3pt}
\bibliography{bib-files/hassan,bib-files/hassan-misc}}
\section{Appendix}
\label{sub:hyper}

  


\end{document}
% \wajih{do not discuss FP/FN in this subsection as I have removed them from the table.}
We conducted experiments to assess how \Sys compares with other systems in terms of detection performance. Initially, we outline our methodology for deploying \Sys on the \darpa E3, E5, and \optc datasets. The E3 dataset comprises various scenarios, including Cadets, Theia, and Trace, each representing logs generated by a single host machine. To evaluate \Sys on E3, we treat each scenario as an individual host. Consequently, in our federated learning approach, we trained local \gnnshort models on each scenario individually. These local models then participated in federated averaging, a process repeated across 10 rounds. Upon completing the training, we evaluated the global \gnnshort model against the attack logs from these E3 scenarios. Similar to E3, the E5 dataset is also divided into different scenarios; however, each scenario comprises three different hosts. For the \optc dataset, we used 10 hosts selected randomly for inclusion in our federated provenance graph learning experiment. Additionally, we conducted an enterprise-level analysis using all 1000 hosts from the \optc dataset, the results of which are detailed in Section~\ref{cost_metric}. For each dataset, we use logs from the benign period to train our system and then evaluate it on the attack logs. The attack logs follow the benign logs in the timeline. For example, the \optc dataset has six days of benign system logs and three days of attack logs. We run \Sys on all days of attack logs to analyze detection performance. For these evaluations, we use the same detection metrics as defined by existing node-level detectors such as \threatrace and \flash.

Table~\ref{summary:benchmarks:large} reveals that \Sys's performance on these datasets is comparable to that of \flash, despite the data heterogeneity, diverse log patterns, and data imbalance contained within each E3, E5, and \optc datasets. This underscores \Sys's capability to maintain robust detection performance amidst such heterogeneity. \kairos's evaluation, based on a coarser time-window granularity compared to the node-level granularity of \flash and \Sys, poses a challenge for direct comparison. Nevertheless, our results remain competitive with \kairos. Beyond detection performance, we also highlight \Sys's qualitative advantages, including its privacy-preserving features and decentralized, scalable operation. These aspects underscore the value of \Sys in contrast to centralized systems, emphasizing its high practicality in real-world deployments. Since \kairos was not originally evaluated on the DARPA E3 Trace, we did not compare \kairos on this dataset for fairness because unlike \flash extensive hyperparameter tuning might be needed for \kairos to produce the best results.

%  \subsection{Resource consumption}

%  We conducted experiments to analyze the resource consumption of the central, utility server, and client-side modules of \Sys. We modeled the resource utilization on a client machine using different batches of audit events of varying sizes. For the central and utility servers, we studied resource consumption by varying the number of clients to understand the demands of federated averaging and semantic vector harmonization. The results, depicted in Figure~\ref{fig:resource}, indicate that \Sys's resource consumption is moderate. Specifically, \Sys can process up to 100,000 audit events simultaneously while consuming less than 900 MB of memory and utilizing less than 20\% of CPU resources. This performance suggests that \Sys does not significantly burden the client machine, especially considering the typically low event throughput on such machines. Additionally, our analysis of the host data in the \optc dataset shows that, on average, each client generates approximately 100,000 audit log events within a three-hour period. For the central and utility servers, the resource usage is minimal, demonstrating that our architecture is scalable and suitable for large organizations with many clients.

%  \begin{figure}[!t]
%   \centering
%   \subfloat[CPU utilization client side.]{\includegraphics[width=0.20\textwidth]{fig/cpu.pdf}\label{cpu_client}}
%   \hfill
%   \subfloat[RAM utilization client side]{\includegraphics[width=0.20\textwidth]{fig/ram.pdf}\label{ram_client}}
%   \hfill
%   \subfloat[CPU utilization central server.]{\includegraphics[width=0.20\textwidth]{fig/cpu_central.pdf}\label{cpu_central}}
%   \hfill
%   \subfloat[RAM utilization central server]{\includegraphics[width=0.20\textwidth]{fig/ram_central.pdf}\label{ram_central}}
%   \hfill
%   \subfloat[CPU utilization utility server.]{\includegraphics[width=0.20\textwidth]{fig/cpu_utility.pdf}\label{cpu_utility}}
%   \hfill
%   \subfloat[RAM utilization utility server]{\includegraphics[width=0.20\textwidth]{fig/ram_utility.pdf}\label{ram_utility}}
%   \caption{Resource consumption of various components of \Sys.}
%   \label{fig:resource}
%   \vspace{-2ex}
% \end{figure}



% \subsection{Enterprise settings evaluation}
% \label{cost_metric}
% We examined the operational costs of deploying \Sys in comparison to centralized solutions like \flash and \kairos for organizational use. Our evaluation concentrates on two primary cost components: network expenses (\(C_{N}\)), and processing costs (\(C_{P}\)).

% \PP{Network Overhead:} We estimate this cost for the \flash and \kairos system using the \optc dataset. Each host within \optc produces approximately 1GB of audit logs daily, equating to nearly one million audit events. For an organization with 1,000 hosts, the total daily log volume would be 1,000 GB. This data volume necessitates transmission over the network to a central server operating the \pids. In contrast, \Sys achieves a significant reduction in these overheads. The only network expenses for \Sys arise from the transmission of the \gnnshort and \wordvec models; the \gnnshort model is roughly 13kb, while the average \wordvec model is 6 MB. Thus, the communication cost with the central server would be 12.70 MB, and for the utility server, it would be 5.86 GB. Ultimately, \Sys results in a 170-fold decrease in communication costs compared to \flash and \kairos.

% \PP{Processing Overhead:} Additionally, \flash processes one million events in about 100 seconds, implying that processing events from 1,000 clients would necessitate approximately 27.7 hours. In contrast, \kairos processes 57,000 events in 11.6 seconds, leading to a processing time of 204 seconds for one million events. Consequently, processing data from 1,000 clients with \kairos would require around 56.6 hours. Compared to existing systems, \Sys processes client logs in a decentralized manner and the total processing time is bounded by the client with the most log data; it will only take approximately 3 minutes to run inference on the complete \optc dataset. The central and utility servers conduct a simple mean operation on the models, taking only a few seconds.

% \PP{Overall Operating costs:} To calculate the daily operational costs of running these systems for a specific organization, we utilized the Google Cloud Platform's (GCP) pricing calculator~\cite{gcp}. The cost to operate \flash under the previously mentioned workload is estimated at \$135, with computing expenses accounting for \$100 and network costs comprising \$35. Given its longer processing duration, the operational cost for \kairos would be twice that of \flash. Compared to these systems, the operational cost of \Sys for 1,000 client machines would be approximately \$10 per day, marking a 13-fold reduction compared to \flash and a 26-fold reduction compared to \kairos.

% Scalability concerns are addressed in this section

\subsection{RQ3: Scalability in Enterprise settings}
\label{cost_metric}
We analyzed the performance of our system in an enterprise setting using the \optc dataset, which includes a large number of host machines. We compared our system to centralized systems such as \flash and \kairos, focusing on important metrics including network and processing overhead. \Fix{We also provide further discussion on scaling FL to massive enterprise networks in Appendix~\ref{state:explosion}.}

\PP{Network Overhead:} We estimate this cost for the \flash and \kairos system using the \optc dataset. Each host within \optc produces approximately 1GB of audit logs daily, equating to nearly one million audit events. For an organization with 1,000 hosts, the total daily log volume would be 1,000 GB. This data volume necessitates transmission over the network to a central server operating the \pids. In contrast, \Sys achieves a significant reduction in these overheads. The only network expenses for \Sys arise from the transmission of the \gnnshort and \wordvec models; the \gnnshort model is roughly \modelsize, while the average \wordvec model is 6 MB. Thus, the communication cost with the central server would be 12.70 MB, and for the utility server, it would be 5.86 GB. Hence, the network latency of model communications in \Sys is minimal compared to centralized systems where the raw system logs need to be sent over the network. Ultimately, \Sys results in a 170-fold decrease in communication costs compared to centralized \pids.  

\PP{Processing Overhead:} Additionally, \flash processes one million events in about 100 seconds, implying that processing events from 1,000 clients would necessitate approximately 27.7 hours. In contrast, \kairos processes 57,000 events in 11.6 seconds, leading to a processing time of 204 seconds for one million events. Consequently, processing data from 1,000 clients with \kairos would require around 56.6 hours. Compared to existing systems, \Sys processes client logs in a decentralized manner and the total processing time is bounded by the client with the most log data; it will only take approximately 3 minutes to run inference on the complete \optc dataset. The central and utility servers conduct a simple mean operation on the models, taking only a few seconds.

\subsection{RQ4: Comparison with existing FL solutions for heterogeneity.}
\label{sec:fedalternatives}

We conducted experiments to evaluate the efficacy of existing federated learning solutions that address data heterogeneity and the non-IID problem. Specifically, we examined FedProx~\cite{li2020federated} and FedOpt~\cite{asad2020fedopt}. FedProx mitigates client heterogeneity by adding a proximal term to the local loss function, penalizing deviations from the global model and thereby reducing the impact of statistical heterogeneity. In contrast, FedOpt uses a server-side optimizer to aggregate updates from distributed clients, enhancing convergence. We compared these techniques with standard FedAvg and the \Sys variant of FedAvg, which incorporates semantic harmonization and categorization-based GNN ensemble learning. We used the \optc, E3, and E5 datasets to conduct this experiment. Table~\ref{fedoptprox} presents the evaluation results. We averaged the results for E3 and E5 across individual scenarios (Cadets, Theia, ClearScope) due to space constraints. Both FedProx and FedOpt outperform FedAvg; however, neither matches the performance of \Sys.

FedProx's penalty on global and local deviations focuses the model on patterns common across all clients, preventing it from learning client-specific patterns. Consequently, using this global model for anomaly detection on individual clients often yields numerous false alarms and poor detection performance. Moreover, while FedOpt can effectively aggregate model updates using an optimizer rather than simple averaging, updates that are highly disparate, a common scenario in system logs due to diverse client activities, cause FedOpt to fail in harmonizing the updates. As a result, it loses information on important benign patterns, leading to low precision and recall.

In contrast, our system employs an ensemble learning framework. First, it categorizes system patterns into standardized bins via process entity categorization. Next, \gnnshort submodels learn the data distribution within each bin. Submodels with similar learned distributions are then averaged, preventing the conflation of unique client patterns and improving precision. Moreover, our semantic vector harmonization reduces heterogeneity in the \gnnshort feature vectors, making the updates less disparate and improving model convergence in comparison to other techniques.


\begin{table}[!t]
  \centering
  \scriptsize
  \caption{Federated averaging algorithms comparison.}
  \setlength{\tabcolsep}{1.6pt}
  \begin{tabular}{cccccccccc}
    \toprule
    \multirow{2}{*}{\textbf{Methods}} & \multicolumn{3}{c }{\textbf{\optc}} & \multicolumn{3}{c }{\textbf{E3}} & \multicolumn{3}{c }{\textbf{E5}} \\
    \cmidrule(r){2-4} \cmidrule(r){5-7} \cmidrule(r){8-10}
    & {\bf Prec.} &  {\bf Rec.} & {\bf F-score} & {\bf Prec.}  & {\bf Rec.} & {\bf F-score} & {\bf Prec.}  & {\bf Rec.} & {\bf F-score} \\
    \midrule

    FedAvg & 0.36 & 0.99 & 0.53  & 0.73 & 0.99 & 0.84 & 0.76 & 0.98 & 0.86 \\
    FedProx & 0.48 & 0.95 & 0.64 & 0.79 & 0.98 & 0.87 & 0.82 & 0.95 & 0.88 \\
    FedOpt & 0.52 & 0.90  & 0.74 & 0.81 & 0.97 & 0.88 & 0.84 & 0.92 & 0.87 \\
    {\bf \Sys} & \TOP & \TOR & \TOF & 0.96 & 0.99 & 0.97 & 0.99 & 0.96 & 0.97 \\
    \bottomrule
  \end{tabular}
\label{fedoptprox}
\end{table}


% \subsection{Robustness against mimicry attacks}
% \label{sec:mimicry}

% We assessed the robustness of our system against the adversarial mimicry attack proposed by Goyal et al.~\cite{goyal2023sometimes} using the E3 dataset. The attack's objective is to generate embeddings for nodes in the attack graphs that are similar to those in the benign graph, thus evading detection. To achieve this, structures from benign nodes are integrated into the attack graph. Figure~\ref{mimicryattack} illustrates the experimental results, with the x-axis representing the number of benign structures added and the y-axis corresponding to anomaly scores for all attack nodes. The findings indicate that our detector maintains robustness against this attack; adding benign structures has minimal effect on the anomaly scores of the attack nodes. The superior robustness of \Sys can be attributed to our process-categorization-based ensemble \gnnshort architecture, wherein each model specializes in understanding the behavior of specific system entities. Thus, when the graph structure surrounding these entities changes, the models can easily detect these alterations.\footnote{In contrast, the \flash system relies on a downstream model using \gnnshort embeddings to detect malicious entities. Generalizing a single model across system entities can cause it to overlook critical details, making it vulnerable to mimicry attacks. As demonstrated by the authors, \flash initially experiences a drop in anomaly scores, increasing the likelihood of attack nodes evading detection, a vulnerability not present in our system.}

% \begin{figure}[!t]
%   \centering
%   \includegraphics[width=0.4\textwidth]{fig/adversarial.pdf}
%   \caption{Adversarial mimicry attack analysis.}
%   \label{mimicryattack}
%   \vspace{-2ex}
% \end{figure}\


% \subsection{Effect of differential privacy on accuracy}

% Differential privacy is a method that can be combined with federated learning to offer protection against inference attacks~\cite{lyu2020threats,nasr2019comprehensive,zari2021efficient} at the cost of detection accuracy. We have analyzed the robustness of our system against these attacks in Section~\ref{privacy}. Here we will examine the impact of differential privacy noise levels on the detection performance of \Sys. Differential privacy achieves model privacy by adding a controlled amount of random noise to the model parameters, which helps in concealing the influence of any individual data point. In our context, differential privacy introduces a parameter called epsilon $\epsilon$ which dictates the intensity of noise added to the local \gnnshort model updates before they are aggregated at the central server for federated averaging, as detailed in Section~\ref{sec:methodology}.

% The parameter $\epsilon$ is crucial; it is inversely related to the amount of noise added -- lower values of $\epsilon$ result in higher noise levels, thereby increasing privacy but potentially degrading the utility of the model. Conversely, a higher $\epsilon$ indicates less noise, which may improve the model's detection capabilities but reduce privacy protection. By adjusting $\epsilon$ during the training phase, we assess the trade-off between privacy and detection performance in the globally trained models across various $\epsilon$ settings. Figure~\ref{epsvsscore} shows that increasing noise strength degrades model utility offering more privacy at the expense of reduced accuracy.

% \begin{figure}[!t]
%   \centering
%   \includegraphics[width=0.25\textwidth]{fig/epsvsscore.pdf}
%   \caption{Effect of differential privacy noise on detection using E3 dataset. Note that we observed similar results on the other datasets.}
%   \label{epsvsscore}
%   \vspace{-2ex}
% \end{figure}


% \begin{figure*}[!t]
%   \centering
%   \subfloat[Anomaly threshold effect.]{\includegraphics[width=0.24\textwidth]{fig/thresh.pdf}\label{thresh}}
%   \hfill
%   \subfloat[Effect of number of categories vs detection performance using E3 dataset.]{\includegraphics[width=0.24\textwidth]{fig/kvsscore.pdf}\label{catgvsscore}}
%   \hfill
%   \subfloat[Federated averaging rounds vs detection performance using E3 dataset.]{\includegraphics[width=0.24\textwidth]{fig/roundsvsscore.pdf}\label{roundsvsscore}}
%   \hfill
%   \subfloat[Effect of number of hosts vs detection metrics using \optc dataset]{\includegraphics[width=0.24\textwidth]{fig/scoresvshosts.pdf}\label{rscoresvshosts}}
%   \caption{Ablation Study of various \Sys components.}
%   \label{ablation}
%   \vspace{-2ex}
% \end{figure*}
