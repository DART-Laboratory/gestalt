\section{Discussion and Limitations}
\label{sec:discussion}

\PP{Adversarial Attacks} \Sys relies on the assumption that Utility server is trusted and does not collude with the recommendation server. Second, \Sys may be brittle to attackers with a large number of malicious clients. These clients can poison the global model by contributing noisy model weights.

\PP{False positives} While \Sys displays a low false positive rate, certain scenarios might lead to false positives, especially in the presence of unobserved benign activities. This is a common challenge with anomaly-based detection systems. Regular model training and adjustment of the alert threshold parameter could alleviate this. Active learning [60], a technique that solicits analyst feedback on ambiguous classifications for model updates, could further reduce false positives. We propose a threat alert triaging method in Section 5 to alleviate the false positive issue. 

\PP{Alert validation} While \Sys expedites alert validation, it’s important to note that its primary objective is threat detection, with alert validation improvement being a secondary benefit. Our approach comes with its inherent limitations. Alert ranking accuracy is closely tied to the precision of threat scores, while the utility of AEGs depends on the correct detection and interpretation of alert relationships - misclassifications can lead to misleading AEGs. Moreover, \Sys’s performance may vary with different operational environments and datasets, requiring continuous fine-tuning. Future work could involve user studies with security analysts to enhance \Sys’s effectiveness further.

\PP{Concept drift} Concept drift, where the data distribution of the underlying system evolves over time, is a potential issue. For instance, with the emergence of new system activities, the patterns learned by \Sys during training might not remain valid. This drift could lead to misclassifications, as new benign behaviors might be mistakenly identified as anomalies. One mitigation strategy for this involves periodic retraining with more recent data to update both the model and the embedding database. Due to its selective traversal, \Sys’s training is efficient, enabling users to periodically retrain their models. However, this approach presents the challenge of preserving the model’s ability to recognize older but still relevant attacks. Unfortunately, no public datasets currently exist to evaluate this strategy’s efficacy. As such, the effective handling of concept drift within \Sys remains a challenge that warrants future research. 